% !TeX spellcheck = en-US
% !TeX encoding = utf8
% !TeX program = pdflatex
% !BIB program = biber
% -*- coding:utf-8 mod:LaTeX -*-


% vv  scroll down to line 200 for content  vv

\PassOptionsToPackage{language=english}{uni-stuttgart-cs-cover}

\let\ifdeutsch\iffalse
\let\ifenglisch\iftrue
\input{pre-documentclass}
\documentclass[
  % fontsize=11pt is the standard
  a4paper,  % Standard format - only KOMAScript uses paper=a4 - https://tex.stackexchange.com/a/61044/9075
  twoside,  % we are optimizing for both screen and two-side printing. So the page numbers will jump, but the content is configured to stay in the middle (by using the geometry package)
  bibliography=totoc,
  %               idxtotoc,   %Index ins Inhaltsverzeichnis
  %               liststotoc, %List of X ins Inhaltsverzeichnis, mit liststotocnumbered werden die Abbildungsverzeichnisse nummeriert
  headsepline,
  cleardoublepage=empty,
  parskip=half,
  %               draft    % um zu sehen, wo noch nachgebessert werden muss - wichtig, da Bindungskorrektur mit drin
  draft=false
]{scrbook}
\input{config}


\usepackage[
  title={A Chatbot for Enhancing Explainability in Smart Homes},
  author={Leon Boppert},
  type=master,
  institute=istesqa, % or other institute names - or just a plain string using {Demo\\Demo...}
  course={Software Engineering},
  examiner={Prof.\ Dr.\ Steffen Becker},
  supervisor={Stefanie Bergner \\ (Robert Bosch Smart Home GmbH)},
  startdate={January 19, 2024},
  enddate={July 19, 2024}
]{uni-stuttgart-cs-cover}
\usepackage{hyperref}
\usepackage{booktabs} % For better looking tables
\usepackage{tabularx} % For adjustable-width columns
\usepackage{makecell} % formating cells in tables
\usepackage{adjustbox}
\usepackage{multirow}

% Hier stehen alle Abkürzungen
\newacronym{er}{ER}{error rate}
\newacronym[plural={RDBMS},shortplural={RDBMS}]{rdbms}{RDBMS}{Relational Database Management System}
\newacronym{llm}{LLM}{Large Language Model}
\newacronym{ai}{AI}{Artificial Intelligence}
\newacronym[plural={Internet of Things},shortplural={IoT}]{iot}{IoT}{Internet of Things}
\newacronym{tp}{TP}{True Positive}
\newacronym{fp}{FP}{False Positive}
\newacronym{tn}{TN}{True Negative}
\newacronym{fn}{FN}{False Negative}
\newacronym{fr}{FR}{Functional Requirement}
\newacronym{nfr}{NFR}{Non-Functional Requirement}

\makeindex

\begin{document}

%tex4ht-Konvertierung verschönern
\iftex4ht
  % tell tex4ht to create picures also for formulas starting with '$'
  % WARNING: a tex4ht run now takes forever!
  \Configure{$}{\PicMath}{\EndPicMath}{}
  %$ % <- syntax highlighting fix for emacs
  \Css{body {text-align:justify;}}

  %conversion of .pdf to .png
  \Configure{graphics*}
  {pdf}
  {\Needs{"convert \csname Gin@base\endcsname.pdf
      \csname Gin@base\endcsname.png"}%
    \Picture[pict]{\csname Gin@base\endcsname.png}%
  }
\fi

%\VerbatimFootnotes %verbatim text in Fußnoten erlauben. Geht normalerweise nicht.

\input{commands}
\pagenumbering{roman}
\Titelblatt

%Eigener Seitenstil fuer die Kurzfassung und das Inhaltsverzeichnis
\deftriplepagestyle{preamble}{}{}{}{}{}{\pagemark}
%Doku zu deftriplepagestyle: scrguide.pdf
\pagestyle{preamble}
\renewcommand*{\chapterpagestyle}{preamble}



%Kurzfassung / abstract
%auch im Stil vom Inhaltsverzeichnis
%\ifdeutsch
 % \section*{Kurzfassung}
%\else
 % \section*{Abstract}
%\fi

%<Short summary of the thesis>

%\cleardoublepage

\section*{Abstract}
%Your abstract should have the following strcuture. This means your asbtract should contain context, problem,... conclusion explicitly as given here. You just need to add your statement in front.
This thesis explores the development and integration of a chatbot system for smart home environments, focusing on enhancing system explainability and user interaction within the Bosch Smart Home ecosystem. As smart home technologies grow increasingly complex, users face challenges in fully understanding and utilizing their capabilities. To address this, we propose a novel approach integrating large language models (LLMs) without extensive fine-tuning to support diverse intents, including device control and data analysis.

The primary objectives of this research are threefold: first, to comprehensively identify the essential components for creating a functional smart home chatbot; second, to push the boundaries of chatbot capabilities by developing and testing a prototype within the Bosch Smart Home app, exploring the feasibility of answering complex questions and managing devices; and third, to enhance system explainability for users, developing features that offer transparent explanations for smart home actions and decisions.

Our methodology involved an iterative development process, combining expert interviews, user surveys, and rigorous evaluation metrics. We implemented a client-server model using Android Studio and Ollama, demonstrating practical deployment potential. The evaluation employed a unique combination of semantic similarity and an metric for function calling accuracy to assess chatbot performance across different model iterations.

Results show that our custom "shllama3instruct" model achieved the best balance between semantic understanding and precise function execution. User studies validated the chatbot's positive impact on system explainability and usability, while also highlighting challenges in user preference and efficiency for certain tasks.

This research contributes an architecture blueprint for integrating LLMs into smart home systems and offers insights into potential standardized frameworks for smart home chatbots. It addresses key industry concerns while advancing academic understanding of complex question-answering in smart home contexts. The findings provide valuable direction for future developments in enhancing user experience and system transparency in smart home technologies, benefiting both technical professionals and business stakeholders in the rapidly evolving smart home sector.
\cleardoublepage

\section*{Kurzfassung}
Diese Arbeit untersucht die Entwicklung und Integration eines Chatbot-Systems für Smart-Home-Umgebungen mit Fokus auf die Verbesserung der Systemerklärbarkeit und Benutzerinteraktion im Bosch Smart Home Ökosystem. Mit zunehmender Komplexität von Smart-Home-Technologien stehen Nutzer vor Herausforderungen, deren Fähigkeiten vollständig zu verstehen und zu nutzen. Um dies zu adressieren, schlagen wir einen neuartigen Ansatz vor, der large language models (LLMs) ohne umfangreiches Fine-Tuning integriert, um verschiedene Absichten, einschließlich Gerätesteuerung und Datenanalyse, zu unterstützen.

Die Hauptziele dieser Forschung lassen sich in drei verschiedene einordnen: Erstens, die wesentlichen Komponenten für die Erstellung eines funktionsfähigen Smart-Home-Chatbots umfassend zu identifizieren; zweitens, die Grenzen der Chatbot-Fähigkeiten zu erweitern, indem ein Prototyp innerhalb der Bosch Smart Home App entwickelt und getestet wird, um die Realisierbarkeit der Beantwortung komplexer Fragen und der Geräteverwaltung zu untersuchen; und drittens, die Erklärbarkeit für Benutzer zu verbessern, indem Funktionen entwickelt werden, die transparente Erklärungen für Smart-Home-Aktionen und -Entscheidungen bieten.

Unsere Methodik umfasste einen iterativen Entwicklungsprozess, der Experteninterviews, Benutzerumfragen und rigorose Bewertungsmetriken kombinierte. Wir implementierten ein Client-Server-Modell mit Android Studio und Ollama, das praktisches Einsatzpotenzial demonstriert. Für die Evaluation wurde eine Kombination aus semantischer Ähnlichkeit und einer Metrik für die Präzision von Funktionsaufrufen verwendet, um die Chatbot-Leistung zu bewerten.

Die Ergebnisse zeigen, dass unser angepasstes "shllama3instruct"-Modell die beste Balance zwischen semantischem Verständnis und präziser Funktionsausführung erreichte. Benutzerstudien bestätigten den positiven Einfluss des Chatbots auf die Systemerklärbarkeit und Benutzerfreundlichkeit, während auch Herausforderungen in Bezug auf Benutzerpräferenzen und Effizienz bei bestimmten Aufgaben hervorgehoben wurden.

Diese Forschung trägt eine Art Architektur-Bauplan für die Integration von LLMs in Smart-Home-Systeme bei und bietet Einblicke in potenzielle standardisierte Frameworks für Smart-Home-Chatbots. Sie adressiert wichtige Branchenbedenken und fördert gleichzeitig das akademische Verständnis komplexer Frage-Antwort-Szenarien im Smart-Home-Kontext. Die Erkenntnisse liefern wertvolle Richtungen für zukünftige Entwicklungen zur Verbesserung der Benutzererfahrung und Systemtransparenz in Smart-Home-Technologien und kommen sowohl technischen Fachleuten als auch anderen Stakeholdern in der sich schnell entwickelnden Smart-Home-Branche zugute.
\cleardoublepage


\section*{Acknowledgment}
I would like to express my deepest gratitude to Sandro Speth, who made this thesis possible by establishing contact with Steffen Becker. I also want to extend my sincere thanks to Steffen Becker for supervising this thesis and for his invaluable support through weekly meetings and regular discussions of my ideas.

A huge thank you to Stefanie Bergner from Bosch Smart Home for her significant contributions and assistance, especially for providing me with contacts to individuals who addressed all my concerns within Bosch Smart Home during my thesis. Her support was further invaluable as she assisted me by recommending me for job applications after the completion of my thesis.

I also want to give special thanks to my partner, Nancy Topalovic, for her unwavering mental support and for making my life easier during stressful phases.

Additionally, I extend my special thanks to everyone who participated in the preliminary interviews and the subsequent user study to explore the developed chatbot. Your contributions were essential to this research. Lastly, I would like to thank the entire team at Bosch Smart Home, with whom I have had the pleasure of working in the past. Your collaboration and support have been greatly appreciated.
\cleardoublepage
% BEGIN: Verzeichnisse

\iftex4ht
\else
  \microtypesetup{protrusion=false}
\fi

%%%
% Literaturverzeichnis ins TOC mit aufnehmen, aber nur wenn nichts anderes mehr hilft!
% \addcontentsline{toc}{chapter}{Literaturverzeichnis}
%
% oder zB
%\addcontentsline{toc}{section}{Abkürzungsverzeichnis}
%
%%%

%Produce table of contents
%
%In case you have trouble with headings reaching into the page numbers, enable the following three lines.
%Hint by http://golatex.de/inhaltsverzeichnis-schreibt-ueber-rand-t3106.html
%
%\makeatletter
%\renewcommand{\@pnumwidth}{2em}
%\makeatother
%
%\setcounter{tocdepth}{2}
\tableofcontents

% Bei einem ungünstigen Seitenumbruch im Inhaltsverzeichnis, kann dieser mit
% \addtocontents{toc}{\protect\newpage}
% an der passenden Stelle im Fließtext erzwungen werden.

\listoffigures
\listoftables

%Wird nur bei Verwendung von der lstlisting-Umgebung mit dem "caption"-Parameter benoetigt
%\lstlistoflistings
%ansonsten:
\ifdeutsch
  \listof{Listing}{Verzeichnis der Listings}
\else
  \listof{Listing}{List of Listings}
\fi

%mittels \newfloat wurde die Algorithmus-Gleitumgebung definiert.
%Mit folgendem Befehl werden alle floats dieses Typs ausgegeben
%\ifdeutsch
%  \listof{Algorithmus}{Verzeichnis der Algorithmen}
%\else
%  \listof{Algorithmus}{List of Algorithms}
%\fi
%\listofalgorithms %Ist nur für Algorithmen, die mittels \begin{algorithm} umschlossen werden, nötig

% Abkürzungsverzeichnis
\printnoidxglossaries

\iftex4ht
\else
  %Optischen Randausgleich und Grauwertkorrektur wieder aktivieren
  \microtypesetup{protrusion=true}
\fi

% END: Verzeichnisse

\mainmatter
\pagenumbering{arabic}

% Headline and footline
\renewcommand*{\chapterpagestyle}{scrplain}
\pagestyle{scrheadings}
\pagestyle{scrheadings}
\ihead[]{}
\chead[]{}
\ohead[]{\headmark}
\cfoot[]{}
\ofoot[\usekomafont{pagenumber}\thepage]{\usekomafont{pagenumber}\thepage}
\ifoot[]{}


%% vv  scroll down for content  vv %%































%%%%%%%%%%%%%%%%%%%%%%%%%%%%%%%%%%%%%%%%%%%%%%%%%%%%%%%%%%%%%%%%%%%%%%%%%%%%%%
%
% Main content starts here
%
%%%%%%%%%%%%%%%%%%%%%%%%%%%%%%%%%%%%%%%%%%%%%%%%%%%%%%%%%%%%%%%%%%%%%%%%%%%%%%


% !TeX spellcheck = en_US
\chapter{Introduction}
%one paragraph per item of the abstract. However, provide more details and do not just copy sentences from the abstract. 
%Additionally, provide  another paragraph and  Make your own \textbf{contribution(s)} explicit ("The contribution of this thesis is...").

% Context / Motivation
In the dynamic field of smart home technologies, users are increasingly turning to sophisticated automation systems to optimize their everyday lives.
As the complexity of these systems continues to grow, there arises a pressing need for innovative solutions that empower users with a deeper understanding of their environments. 
Picture a user with a huge amount of automations in their smart home, occupying with complex questions like, "Why did I have higher heating costs this winter?" or "What was the intention behind the light just turning on in the kitchen?"
While technology-interested users may attempt to answer such questions themselves through self-analysis of their smart home systems, the complexity of such tasks can be time-consuming and may not be simply feasible for everyone.
These scenarios encapsulate the motivation behind this thesis, which centers on the development of a chatbot for smart home systems and home automation.
The primary focus is increasing system explainability and addressing the unique challenges users encounter in analyzing the intricacies of their smart homes.

Furthermore, the implementation of a chatbot functionality holds the potential for additional benefits, including a reduction in support requests as users gain autonomy in troubleshooting which also is available at any time. 
By storing and analyzing user requests, the system could also provide insights into the needs and interests of users and thus influence future developments. 
Moreover, the chatbot could act as a valuable tool in identifying unintended behavior or bugs within the smart home system, contributing to a more robust and user-friendly technology landscape.

% Problem
%-users are confronted with a lack of tools that can effectively provide insights into the rationale behind system actions
%-inconsistencies in the "decisions" of smart home devices --> user and developers want to understand
%-absence of a reference architecture for a chatbot tailored to answer complex questions in a smart home environment
When it comes to smart home technology, users face a variety of difficulties that draw attention to important issues that call for creative solutions.  First and foremost, users are confronted with a lack of tools capable of effectively providing insights into the rationale behind system actions.
Inconsistencies in the "decisions" of smart home devices extend the problems even further, as developers and users can struggle to comprehend the underlying logic of actions that are executed, with the added challenge of distinguishing whether the observed behavior is indicative of a potential bug or a result of a specific user configuration.

An hurdle for the planned development throughout this thesis is the lack of a reference architecture designed especially for chatbots to answer complicated queries inside a smart home system.
Therefore it has to be analyzed if there exist related work on chatbots which answer complex questions in a similar way to potentially inspire this work.

Another problem for the thesis is the lack of available data in the bosch smart home system in which the chatbot should be integrated.
This deficiency manifests in several dimensions, beginning with the cost-intensive nature of cloud services and the consequential limitations imposed by a surge in user requests that could occur in requests to a chatbot. 
Compounding this issue, smart home devices have constrained computing resources due to their compact sizes which may make the accumulation of data about the system difficult or slow. 
The integral role of a knowledge base for chatbot functionality is the center of these problems, as it heavily relies on data. 
In the current state of the Bosch Smart Home System, although valuable data concerning device resources and system logs exists, accessibility remains restricted since logs only get available when users send a support request.
The data of individual smart homes would represent an excellent source for chatbot functionality, including resource metrics such as CPU and RAM usage and logs detailing system actions.
However, the challenges persist, requiring careful consideration of data availability, structuring, and integration into the chatbot's architecture to form a reliable knowledge base. 
These data-related challenges come together to the pressing need for innovative solutions to enhance the data accessibility and usability of smart home systems.

%    - cost-intensity through cloud services and huge amount of requests 
%   - limited ressources due to device size
%    - A knowledge base is a crucial part of chatbots which is based on data--
%    - state in Bosch Smart Home System is that data about devices is available but not accessable without a request to the user support.
%        --> this data consists of ressource data (cpu usage, ram usage, ...) and logs (containing system actions like turning devices on or off)
%    - depending on the architecture of the chatbot the data has to be available and possibly be structured to form a reliable knowledge base


% Objective
The objectives of this thesis are multifaceted, aiming to find important aspects in developing a functional chatbot tailored for answering complex questions in the area of smart homes. The first goal is to comprehensively grasp the essential components needed for the creation of a working chatbot. This involves identifying requirements and evaluating tools suitable for building a chatbot, with a specific focus on the feasibility of utilizing open-source options to answer if they effectively can contribute to such a project.

Subsequently, the research attempts to push the boundaries of chatbot capabilities by exemplary developing a chatbot into the Bosch Smart Home app and testing how far such a system can go in answering complex questions. The general question here is whether it is even plausible to develop a chatbot for the mentioned application. This objective involves finding and assessing the technical limitations and potential breakthroughs in creating an adaptable chatbot for this specific domain.

Finally, by providing explainability for users, the thesis seeks to satisfy the user-centric aspect of smart home technologies. This objective involves developing features within the chatbot that not only answer questions but also offer transparent and understandable explanations regarding the actions and decisions made by the smart home system. This user-focused approach seeks to enhance the overall user experience and potentially bring benefits as comprehension and trust to users about their own smart home. 

%  Method:
\begin{figure}[b]
\centering
\includegraphics[width=0.95\textwidth]{graphics/iterative-design.png}
\caption{Visualization of the iterative development process.}
\label{fig:iterative-design}
\end{figure}
The research methodology for this thesis is splitted into two phases, with the first one focusing on a comprehensive literature review.
In this initial step existing concepts and building blocks for chatbot development will be gathered to provide a foundation for understanding how to develop an intelligent chatbot.
Additionally, the literature review should provide prior work that developed chatbots with similar functionalities even if in other domains.
This analysis should help to identify insights and best practices transferable for the implementation of a chatbot tailored to smart home applications.

The second key phase of the methodology involves an iterative development process to create an example architecture and chatbot prototype. 
This process is shown in \cref{fig:iterative-design}.
During the planning stage, the research defines useful intents and requirements essential for the chatbot's functionality. 
This stage also includes the identification of valuable tools and technologies, along with strategies for making the necessary data accessible to the chatbot.

The subsequent implementation phase involves the analysis and design of a new implementation based on the identified intents and requirements. This step aims to construct a chatbot architecture that aligns with the specific needs of answering complex questions within the smart home domain.

Lastly, the methodology includes a robust evaluation process to assess the perfromance of the chatbot prototype. 
This evaluation is also part of the iterative process and focuses on answering how effectively the chatbot addresses the predefined intents and requirements, providing valuable insights into its performance and areas for potential refinement. 

% todo Result
The potential result of the thesis is the development and evaluation of a chatbot for smart home systems, enhancing system explainability and user interaction. 
The research outcome offers valuable insights into the feasibility, challenges, and performance of such a chatbot, providing a foundation for future chatbots with similar use-cases.

% Conclusion
In conclusion, this thesis investigates how far a chatbot can go in addressing complex user queries to provide enhanced system explainability and user-centric functionalities.
The research outcomes underscore the potential of such chatbots while simultaneously unveiling challenges that can occur in the development process.
The findings showcase the ability of the chatbot to enhance user understanding and engagement within the smart home domain. 
The study not only contributes to the field of smart home technologies but could also act as a contact point for future innovations, where chatbots are used for intuitive and comprehensible interaction between users and their systems.

% Contributions [explicit]
\section*{Explicit Contributions}
The thesis contributes to the field of smart home technologies by developing a chatbot for home automation systems, exemplary into the Bosch Smart Home system, focusing on enhancing system explainability and addressing user-centric functionalities. 
It provides a novel exploration of chatbot architectures, tools, and methodologies that could be used for answering complex questions within smart home environments. 
The research provides insights into potential challenges of implementing such a chatbot, assesses its capabilities, and offers recommendations for future developments. 
Ultimately, this thesis aims to give users a deeper understanding of their smart homes and enhance transparency and usability of home automation systems.
\section{Objective}

\section{Methodology}

\section*{Thesis Structure}
Here, give an overview of your thesis structure.
\begin{description}
\item[Chapter~\ref{chap:evaluation} -- \nameref{chap:evaluation}:] Here, we provide...
\item[Chapter~\ref{chap:conclusion} -- \nameref{chap:conclusion}] We conclude our thesis ...
\end{description}

% !TeX spellcheck = en_US

\chapter{Foundations and Related Work}
\label{chap:ch2}
\label{chap:foundation}

In the exploration of chatbot technology and its applications, this section delves into foundational aspects and related work that form the basis of effective chatbot development. 
The foundations include concepts such as smart homes, chatbot types, and building blocks, providing insights into the underlying principles and methodologies. 
As the foundation is established, the discussion extends to related work, examining existing literature on chatbots in smart homes and their applications in complex scenarios. 
By understanding these foundational elements and existing research, the subsequent sections aim to contribute novel insights and advancements in the field of chatbot development.

\section{Foundations}
\label{sec:foundation}
In this subsection, the focus are the specific foundations that are crucial for the development of chatbots.
Including key elements for this thesis such as a definition for a Smart Home and Natural Language Understanding which enables chatbots to understand a users input message.
Other elements such as a general architecture and typical building blocks for chatbots are also presented.

\subsection*{Smart Home} 
A smart home devices main aspect is that its original functionality is augmented through network capabilities \cite{schiefer_smart_2015,balakrishnan_smart_2018} and should enhance especially comfort\cite{matsui_information_2018, balakrishnan_smart_2018} but also security\cite{balakrishnan_smart_2018} or energy efficiency\cite{matsui_information_2018, balakrishnan_smart_2018} for example. 
Therefore a smart home consists of such devices and provides additional infrastructure like an app to control the devices. 
These infrastructures often allow automation of the devices.
Since its functionality is based on connected devices and often sensors it is often named in the context of the Internet of Things (IoT) \cite{atzori_internet_2010}. 

\subsection*{Chatbot Types and Classifications}
% lehmann
Based on Lehmann \cite{lehmann_chatbot-guide_2021} there are three main types of chatbots: decision-tree-based, keyword-based, and advanced context-aware bots. 
Decision-tree-based bots follow a predefined flowchart in response to user queries, often identified by menus and buttons. 
Keyword-based bots recognize specific keywords, making decisions and providing responses based on internal knowledge. 
Advanced context-aware virtual assistants engage in free-flowing conversations, learning from interactions and offering versatile responses.

% adamopoulou
Additionally, according to Adamopoulou and Moussiades\cite{adamopoulou_overview_2020} chatbots can be classified based on their \textbf{knowledge domain}, with open domain bots discussing general topics and closed domain bots focusing on specific knowledge domains. 
In terms of \textbf{service}, interpersonal chatbots act as information intermediaries, offering communication services like restaurant or flight booking. 
Intrapersonal chatbots, existing within the user's personal domain, function as companions, understanding users like humans. 
Inter-agent chatbots, omnipresent in nature, facilitate inter-chatbot communication. 
\textbf{Goal-based} classification includes informative chatbots providing stored information, chat-based bots conversing like humans, and task-based bots performing specific functions intelligently. 
\textbf{Input processing and response generation} methods vary, from rule-based models relying on predefined rules to retrieval-based models using APIs, and generative models employing machine learning techniques. 
\textbf{Human-aided} chatbots incorporate human computation for flexibility and robustness.
The \textbf{build method} distinguishes between open-source platforms allowing intervention and closed platforms acting as black boxes but providing immediate access to advanced technologies, often found in large companies.

\subsection*{Chatbot Building Blocks}
In this subsection building blocks and architectural insights that can be identified in different literature are collected and explained. 
It is mostly based on Lehmann \cite{lehmann_chatbot-guide_2021} and \cite{adamopoulou_overview_2020} if not mentioned otherwise.

\subsubsection{Natural Language Understanding (NLU) and Machine Learning}
NLU, a subset of Natural Language Processing (NLP), is essential for chatbots.
It involves training algorithms to understand and process natural language, bridging the gap between human language and artificial intelligence. 
NLU is critical for successful chatbot operation, enabling the machine to comprehend user intents and contexts.

Machine learning and NLP also play roles in the evolution of virtual assistants. 
Machine learning algorithms, based on training data, build statistical models and autonomously improve over time.
NLP focuses on processing and understanding natural language, incorporating elements from computer science, AI, linguistics, and data mining. 
These technologies empower chatbots to interpret complex human language, allowing for more sophisticated and context-aware interactions.

\subsubsection{Intents}
Chatbots operate with three key components – Utterances, Intents, and Entities. Intents represent different user intentions that a chatbot anticipates, each consisting of various utterances (training phrases) and one or more responses. Utterances are potential user expressions or example questions, and entities are real-world objects mentioned in a sentence. This structure facilitates the interpretation of natural language, enabling chatbots to recognize user intents and associated entities for effective communication.

Chatbots are built on different intents, each with various utterances and responses. 
The training involves teaching the bot to recognize user intents based on a set of training phrases. 
A Natural Language Processing Engine interprets natural language, transforming it into structured language, and identifies entities within sentences to enhance contextual understanding.

There are also narrow work that regard intent-based chatbots as a separate species \cite{luo_critical_2022}. 
These then typically deal with the technique used to generate responses to ultimately categorize the chatbots.

\subsubsection{Knowledge Bases}
Knowledge bases serve as a crucial component for chatbots, providing them with the necessary information to respond intelligently to user queries. 
These databases store domain-specific data, facts, and contextual information that enable chatbots to understand and address user requests accurately.

\subsubsection{General Chatbot Architecture}
In general, the chatbot architecture by Adamopoulou and Moussiades\cite{adamopoulou_overview_2020} which can be seen in \cref{fig:chatbot-architecture-general} consists of components such as the Language Understanding Component, which interprets user requests and identifies intentions and associated information, and the Dialogue Management Component, which keeps track of the conversation context, processes clarifications, and asks follow-up questions. After understanding the user's request, the chatbot executes actions or retrieves data from its Knowledge Base or external resources, and the Response Generation Component generates natural language responses based on the intent and context information.


\iffalse
\subsubsection{A chatbot taxonomy}
\begin{figure}[h]
\centering
\includegraphics[width=0.98\textwidth]{graphics/chatbot-taxonomy-raymond.png}
\caption{An example taxonomy of chatbots \\Source: Luo et al. \cite{luo_critical_2022}}
\label{fig:chatbot-taxonomy}
\end{figure}
\newpage
\fi


\subsection*{Prototype and Iterative Development}

This subsection is based on Sommerville \cite{sommerville_software_2011} according to who "iterative development of the prototype is essential".
The process for developing a prototype is shown in \cref{fig:prototype-process} and summed up consists of planning, developing and evaluating.
Another iterative process is the extreme programming release cycle which is shown in \cref{fig:extreme-prog-cycle} and consists of user stories that are selected and developed throughout one cycle.
Based on this concepts and needs for our chatbot prototype we built the iterative development cycle that is shown in \cref{fig:iterative-design} which could be seen as a combination of the both processes presented in this subsection.
It is useful to start projects like this thesis with a prototype with thought out case \cite{lehmann_chatbot-guide_2021}.

\begin{figure}[t]
\centering
\captionsetup{justification=centering}
\includegraphics[width=0.98\textwidth]{graphics/chatbot-architecture-general.png}
\caption{An example architecture of chatbots in general \\Source: Adamopoulou and Moussiades \cite{adamopoulou_overview_2020}}
\label{fig:chatbot-architecture-general}
\end{figure}

\begin{figure}[h]
\centering
  \begin{subfigure}{.7\textwidth}
    \includegraphics[width=\textwidth]{graphics/prototype-dev.png}
    \caption{Prototype development process}
    \label{fig:prototype-process}
  \end{subfigure} \hfill
  \begin{subfigure}{.63\textwidth}
    \includegraphics[width=\textwidth]{graphics/extreme-programming-release-cycle.png}
    \caption{Extreme programming release cycle}
    \label{fig:extreme-prog-cycle}
    \end{subfigure}
  \caption{Visualized development processes \\ Source: Sommerville \cite{sommerville_software_2011}}
  \label{fig:dev-processes}
\end{figure}

\newpage
\section{Related work}
In addition to discussing about foundations, you need to discuss about why the problem you are tackling has been insufficiently solved before. To do so, you need to mention related work and discuss what is the difference between your work and those related works.
This section presents literature that is related to this thesis.
While chatbots are ever known to be used in areas like customer service it is interesting to see work on how chatbots are used in smart homes or for complex scenarios like managing containerized networks.
Some literature has the nice side effect that it also presents the architecture of the developed system and can inspire the architecture of the chatbot that should be developed in this thesis.

\subsection*{Chatbots for Smart Homes}
Various literature exists that contains some kind of chatbot in the context of Smart Homes.
Most of these chatbots perform the same task as nowadays common Voice assistants as for example the Google Assistant\footnote{\href{https://assistant.google.com/}{assistant.google.com}} which is capable of understanding written request but also transforming speech into written requests which also includes managing smart devices added to Google Home\footnote{\href{https://home.google.com/intl/de_de/the-latest/}{home.google.com}}, an app for managing a smart home.

Baby et al.\cite{baby_home_2017} presented an approach for a chatbot that can control multiple smart devices in a smart home.
The developed prototype is capable of answering simple questions that regard the devices or change variables of the devices.
It is able to answer questions like "What is the temperature in room 1" or "Set the temperature to 19 degrees celsius".
The architecture includes multiple building blocks that were explained in \cref{sec:foundation}: An NLP Pipeline which in the end identifies the intent of a message and matches an action to.
The pipeline can be seen in \cref{fig:chatbot-pipeline-baby}
The intents and pipeline of this chatbot could inspire the inital iteration of our prototype. 
\begin{figure}[h]
\centering
%\captionsetup{justification=centering}
\includegraphics[width=0.66\textwidth]{graphics/baby2017chatbot.png}
\caption{A simple NLP pipeline for a chatbot \\Source: Baby et al. \cite{baby_home_2017}}
\label{fig:chatbot-pipeline-baby}
\end{figure}

Another work developed and connected a chatbot to the facebook (today called meta) messenger for controling smart home devices fans and lights but also gas leakage detection or humidity monitoring \cite{ahmed_smart_2020}.
Intents are not clearly defined or at least not stated but it seems to be a simple decision-tree-based chatbot.
This is probably due to focus of the work laying more in the systems architecture and a limited amount of intents.

A different work explored the area of collaboratively teaching where a focus was set on mitigate malicious activities \cite{chkroun_safe_2021}.
The presented chatbot called Safebot is intended as an extension to smart home assistants and is able to communicate when it does not know the answer the a request and can be taught afterwards.
Also, users could notify the bot that an response was not appropriate resulting in also teaching it.
The learning is purely based on natural languages in contrast to chatbots that for example use knowledge bases with structured data.

\subsection*{Chatbots for complex Scenarios}
Various chatbots for many different use cases exist. 
As the intention of this work is to test how far a chatbot can go in answering complex queries to a smart home the question aligns if approaches exist for answering (complex) questions in a closed domain based on available data.
An example for this is the work of Ait-Mlouk and Jiang \cite{ait-mlouk_kbot_2020} which approached to develop a knowledge graph based chatbot that can find information in linked data through NLU.
The researches in this work address challenges like understanding many different queries (and intents) and making the use of multiple languages and knowledge bases available.
The system developed is able to be extended with new domains.
The architecture of it can be seen in \cref{fig:kbot}.
While it is too complex to explain in detail, it processes queries into intents and entities (through Named Entity Recognition) which in combination make it possible to retrieve information from connected knowledge bases by transforming it into queries and selects a response based on a knowledge graph.

\begin{figure}[h]
\centering
%\captionsetup{justification=centering}
\includegraphics[width=0.94\textwidth]{graphics/KBot-architecture.png}
\caption{Detailed Architecture of KBot \\Source: Ait-Mlouk and Jiang \cite{ait-mlouk_kbot_2020}}
\label{fig:kbot}
\end{figure}

Another research presented a chatbot that can create and manage a containerized network, thus acting as a workflow manager \cite{jasinski_chatbot-based_2023}.
It enables less human involvement by computing requirements written by users.
This leads to a decreasing need for users to have extensive knowledge of the underlying tools to be able to setup such networks and also makes the management of them easier in general.
This can be mapped to this thesis where one aim is to enable even users with less technology knowledge to analyze complex scenarios in their smart home and to make the analysis of those easier in general.

A different work \cite{carlander-reuterfelt_jaicob_2020} approached to develop a chatbot that helps to gain knowledge about Data Science and Machine Learning.
The architecture includes information retrieval from a knowledge base, parsing the received document and answering a query based on the search in the knowledge base.
It also includes a "Small Talk Module" which improves users satisfaction with the system and increase the overall interest in chatting with it.

%noch etwas Fülltext
%\blinddocument

\chapter{Methodology}
\label{chap:method}
This chapter outlines the research methodology employed in this thesis, focusing on the scientific methods used to address the research objectives.

The research methodology of this master's thesis is a comprehensive, mixed-methods approach that incorporates elements of design science and exploratory and empirical research.
This approach is well-suited for a thesis in the field of smart home technologies which are deeply connected with human-computer interaction while we at the same time also take a look at the architecture and building blocks of the intended chatbot.
Due to its structured nature, design science enables an iterative development of the chatbot and the solving of specific problems and therefore is commonly used in software engineering research.
On the other hand the explorative empirical approach aids in understanding user requirements and gathering real-world data to inform the design process and in the end collecting and analyzing evidence that is observable and measurable.

\paragraph{Methods Used:}
%Literature Review including the topics: classification and types of chatbots, typical building blocks used in chatbots, function calling in language models and similar chatbot research, for example smart home chatbots for simple tasks or chatbots for complex scenarios like using multiple knowledge bases, evaluation techniques used for similar language model based chatbots
%iterative
%The chatbot was developed iteratively, as visualized in \cref{fig:iterative-design} of the Introduction chapter. Each iteration involved planning, implementation, testing, and evaluation phases to refine the chatbot's functionality based on performance. Initially, the approach was to add new intents with each iteration. However, the focus shifted to exploring how to develop the first intents effectively and iteratively improving their addressing. This adjustment allowed for a more thorough understanding and enhancement of the initial functionalities.
%preliminary interviews with experts of different domains to gather requirements and ideas for the chatbot development. This qualitative data informed the initial design and development stages.
%survey for gathering example user inputs for the chatbot. Resulting from the preliminary interviews we conducted a survey to gather example user inputs for the chatbots which were essential for developing and also the evaluation
%prototype implementation
%A prototype of the chatbot was designed and implemented within the Bosch Smart Home system, utilizing a client-server architecture. The technology stack included open-source tools to support the development and integration processes.
%study design:
%model performance mostly quantitative but also qualitative analysis of what did work out what did not and why, including the building of an evaluation dataset partly based on the  
%quantitative and qualitative analysis through an experiment: user study with tasks to complete with the chatbot, questionnaire and semi-structured interviews.

A comprehensive literature review was conducted, covering several key areas relevant to chatbot development and smart home technologies. This review explored the classification and types of chatbots, typical building blocks used in their construction, and the emerging technique of function calling in language models. Additionally, similar chatbot research was examined, including studies on smart home chatbots for simple tasks and more complex scenarios involving multiple knowledge bases. The review also investigated evaluation techniques used for language model-based chatbots, providing a solid foundation for the evaluation methodology employed in this study.

The development process followed an iterative approach, as illustrated in \cref{fig:iterative-design} of the Introduction chapter. Initially, the strategy was to introduce new intents with each iteration. However, as the project progressed, the focus shifted towards developing the initial intents more effectively and iteratively improving their implementation. This adjustment enabled a more thorough understanding and enhancement of the core functionalities.

To gather requirements and ideas for the chatbot development, preliminary interviews were conducted with experts from various domains. This qualitative data was instrumental in informing the initial design and development stages, ensuring that the chatbot's features aligned with real-world needs and expectations.

Following the insights gained from the preliminary interviews, a survey was conducted to collect example user inputs for the chatbot. These user inputs were crucial not only for the development process but also for the subsequent evaluation phase. By incorporating real user language and expectations, the chatbot could be designed to better meet user needs and preferences.

A prototype of the chatbot was designed and implemented within the Bosch Smart Home system, utilizing a client-server architecture. The technology stack was carefully selected to include valuable open-source tools.

The evaluation of the chatbot employed a mixed-methods approach, combining quantitative and qualitative analyses. The model performance was assessed primarily through quantitative metrics, but also included a qualitative analysis of successful and unsuccessful outcomes, examining the reasons behind these results. An evaluation dataset was constructed, partly based on the user inputs gathered from the survey, to ensure a comprehensive and realistic assessment of the chatbot's capabilities.

To gain insights into the user experience and overall effectiveness of the chatbot, a user study was conducted. This study comprised a set of tasks for participants to complete using the chatbot, followed by a questionnaire to gather quantitative data on user satisfaction and perceived usability. Additionally, semi-structured interviews were conducted to collect rich, qualitative data on user experiences, preferences, and suggestions for improvement.
% !TeX spellcheck = en_US

\chapter{Concept}
\label{chap:concept}

In this chapter, we describe the intents defined for the smart home system chatbot. These intents aim to provide users with a versatile and intuitive interface for interacting with their smart home devices. The primary focus is on ensuring the chatbot can understand and respond to a wide range of user inputs, offering functionalities that surpass current smart home solutions.

\section{Intent Engineering}

Intents are fundamental to the functionality of the chatbot. Each intent represents a specific user request or command that the system needs to understand and act upon. The design of these intents considers the limitations of existing voice assistants and aims to provide a more flexible and user-friendly experience.

\subsection{Iteration 1: Basic Intents}

\paragraph{Providing Device Status}

\begin{itemize}
    \item \textbf{Intent Name:} GetDeviceStatus
    \item \textbf{Examples:}
    \begin{itemize}
        \item "What is the temperature in the living room?"
        \item "What is the status of the thermostat in the living room?"
        \item "Is it warm in the living room?"
    \end{itemize}
    \item \textbf{Entities:}
    \begin{itemize}
        \item DeviceType (e.g., thermostat, lights)
        \item Room (e.g., living room, bedroom)
    \end{itemize}
    \item \textbf{Action/Response:} Providing information about the specified device in the given room, such as the temperature and power status of a thermostat.
\end{itemize}

This intent aims to provide a more conversational approach to querying device statuses. Unlike existing solutions that require specific device names, our chatbot can understand various formulations, allowing users to ask for the temperature without needing to specify the exact device name or its location.

\paragraph{Changing Device Status}

\begin{itemize}
    \item \textbf{Intent Name:} SetDeviceStatus
    \item \textbf{Examples:}
    \begin{itemize}
        \item "Set the temperature to 22 degrees in the living room."
        \item "Turn off the lights in the bedroom."
        \item "Make it cooler in the kitchen."
    \end{itemize}
    \item \textbf{Entities:}
    \begin{itemize}
        \item DeviceType (e.g., thermostat, lights)
        \item Room (e.g., living room, bedroom)
        \item DesiredStatus (e.g., temperature, on/off state)
        \item DesiredValue (e.g., specific temperature, on/off)
    \end{itemize}
    \item \textbf{Action/Response:} Executing the specified action to set the desired status of the mentioned device in the given room, such as adjusting the temperature for a thermostat or turning lights on or off.
\end{itemize}

This intent enhances user experience by allowing natural language commands to control devices. The chatbot interprets a wider range of user commands, enabling users to control their devices more intuitively and without needing to remember specific device names.

\subsection{Iteration 2: Intermediate Intents}

\paragraph{Assistance for Creating Automations}

\begin{itemize}
    \item \textbf{Intent Name:} CreateAutomation
    \item \textbf{Examples:}
    \begin{itemize}
        \item "Set up an automation for turning off lights at 10 PM."
        \item "Create a rule to adjust thermostat settings when I leave home."
        \item "Can you help me with automating my smart blinds?"
        \item "Notify me if any windows are open."
    \end{itemize}
    \item \textbf{Entities:}
    \begin{itemize}
        \item DeviceType (e.g., lights, thermostat, blinds)
        \item TriggerEvent (e.g., time-based, occupancy, temperature change)
        \item Condition (optional, e.g., specific temperature threshold)
        \item Action (e.g., turn off, adjust settings)
        \item Location (optional, e.g., living room, bedroom)
    \end{itemize}
    \item \textbf{Action/Response:} Assisting the user in defining and setting up a smart home automation, including specifying the devices involved, the triggering event, any conditions, and the desired actions. The chatbot may also provide suggestions for common automation scenarios.
\end{itemize}

Creating automations is a common use case in smart home systems. This intent aims to streamline the process, allowing users to set up complex automations through simple conversational interactions, thus reducing the need for technical knowledge or precise phrasing.

\paragraph{Interpret the Device Control}

\begin{itemize}
    \item \textbf{Intent Name:} DeviceControlInterpretation
    \item \textbf{Examples:}
    \begin{itemize}
        \item "Why did the lights in the bathroom turn on just now?"
        \item "Can you explain the reason for the thermostat adjusting the temperature?"
        \item "What triggered the blinds to open in the living room?"
    \end{itemize}
    \item \textbf{Entities:}
    \begin{itemize}
        \item DeviceType (e.g., lights, thermostat, blinds)
        \item Location (optional, e.g., bathroom, living room)
        \item Action (e.g., turn on, adjust temperature, open)
        \item TriggerSource (e.g., automation, manual activation)
        \item Timestamp (optional, for specifying a time reference)
        \item Reason (optional, e.g., reason for an automation that the user provided when creating it)
    \end{itemize}
    \item \textbf{Action/Response:} Providing an explanation for recent smart home device actions. The chatbot interprets the cause of device events, distinguishing between automation-driven events and those triggered manually by the user. It may also consider time-based context when explaining device actions.
\end{itemize}

This intent addresses a gap in current systems by explaining the reasons behind device actions. It improves transparency and user trust in smart home systems by providing clear explanations for automated and manual device actions.

\subsection{Iteration 3: Complex Intents}

\paragraph{Analyzing Energy Consumption}

\begin{itemize}
    \item \textbf{Intent Name:} AnalyzeEnergyConsumption
    \item \textbf{Examples:}
    \begin{itemize}
        \item "Can you analyze the energy consumption in my home?"
        \item "Provide insights into power usage over the last week."
        \item "How can I optimize energy consumption in the living room?"
    \end{itemize}
    \item \textbf{Entities:}
    \begin{itemize}
        \item AnalysisType (e.g., overall consumption, specific devices)
        \item TimeFrame (e.g., last week, last month)
        \item Room (e.g., living room, kitchen)
    \end{itemize}
    \item \textbf{Action/Response:} Generating a detailed analysis of energy consumption based on the specified parameters. It includes insights into overall energy usage, specific device contributions, and recommendations for optimizing energy consumption in the specified room or timeframe.
\end{itemize}

Energy consumption analysis is a valuable addition to smart home capabilities. This intent provides users with actionable insights into their energy use, helping them to make informed decisions about energy efficiency and cost savings.


%% Auslagern in eigenes chapter??? %%
\section{Preleminary Interviews to Gain Knowledge, Requirements and Ideas}
Preceding to further actions Interviews with a few individuals were conducted to gain knowledge, requirements and ideas for this thesis.
All of the four interviewed were from different domains: One is a researcher and has knowledge in \gls{ai} technologies or more precisely about \glspl{llm} while the three other interviewees are from Bosch.
One of them is working in \qq{AskBosch} but also has expertise in the Smart Home and \gls{iot}. 
The other two are both from Bosch Smart Home where one is a Software Developer and the other a Product Developer.
Therefore a diverse group of interviewees has been formed.

The interview was designed to be open but some guided questions were formulated which were selected based on the domain of the interviewed since not all questions met every present domain.
For example it would have not make sense to ask a Smart Home Software Developer about recent tools or the architecture of \gls{llm}-based applications.
In the following all guided questions are listed:

\begin{enumerate}
    \item \textbf{Feasibility and Key Considerations} \\
    What are your thoughts on the feasibility of a smart home chatbot for providing explainability and enhancing user experience?
    \item \textbf{Interesting Intents}
    From your perspective, what could be interesting intents for a smart home chatbot to address complex user queries?    
    \item \textbf{Selected Intents}
    What is your opinion on the specific intents that have been selected for the chatbot development?    
    \item \textbf{Tools/Technologies/Models}
    In your research experience, what tools, technologies, or models would you recommend for developing a chatbot tailored for smart home applications?
    Would you chose a large language model and input context and user request into it or rather a NLP pipeline that chooses further actions based on the user request?   
    \item \textbf{Data Organization}
    How would you suggest organizing and feeding data to the chatbot, considering the complexity of smart home scenarios?    
    \item \textbf{Common Issues/Pitfalls}
    Based on your expertise, what are the common issues or pitfalls researchers may encounter when developing chatbots for specialized domains like smart homes?
    \item \textbf{Evaluation of Success}
    How would you propose evaluating the success of a smart home chatbot, especially in terms of providing explainability and user-centric interaction? 
    \item \textbf{Data Sources for the Chatbot}
    Can you identify potential sources from which data for the chatbot could be extracted to address user queries about the smart home?
    \item \textbf{Accessibility of Data}
    Do you foresee any challenges or limitations in accessing the identified data sources for the chatbot development?
           
\end{enumerate}

\section{Requirements}

\section{Idea}

\section{Architecture}

%\section{further parts of the prototype - NLP Pipeline, Integration into the Bosch SH App,...}

%\blinddocument

% !TeX spellcheck = en_US
\chapter{Implementation}
\label{chap:implementation}
This chapter details the technical realization of the Smart Home Chatbot system, translating the conceptual framework into a functional application. We begin with an overview of our technology stack, which combines Android Studio for client-side development and Ollama for server-side language model hosting.

The chapter is structured to distinctly cover both server-side and client-side components. For the server, we explore the setup process, challenges faced, and model customization techniques. On the client side, we delve into the Android application's development, covering data management within the Bosch Smart Home ecosystem, user interface design, and message handling.

We then illustrate the interaction flow between user, client, and server, demonstrating how these components collaborate to process queries and generate responses. The chapter concludes by addressing key challenges encountered during development, such as multilingual support and historical data limitations, along with their implemented or proposed solutions.

This comprehensive overview provides readers with a thorough understanding of the system's technical architecture and the rationale behind critical design decisions.



\section{Technology Stack}
The implementation of the chatbot system leverages a diverse set of technologies, each chosen for its specific capabilities and compatibility with the existing Bosch Smart Home ecosystem. Figure \ref{fig:techstack} illustrates the technology stack overlaid on the base architecture.

Android Studio\footnote{\url{https://developer.android.com/studio}} serves as the primary integrated development environment, facilitating the extension of the existing Bosch Smart Home Android application. The client-side development utilizes Java\footnote{\url{https://www.java.com/}} programming language in conjunction with Android 14 SDK\footnote{\url{https://developer.android.com/about/versions/14}}, ensuring compatibility with the latest Android features and optimizations.

To enhance the user interface and manage the chat functionality efficiently, the implementation incorporates modern Android components. RecyclerView is employed for rendering the message exchange between the user and the chatbot, providing smooth scrolling and efficient memory usage. Concurrency tools, specifically ExecutorService and CompletableFuture, are utilized to handle gls{api} calls in the background, ensuring a responsive user interface while managing asynchronous operations.

On the server side, Ollama v0.1.47\footnote{\url{https://ollama.com/}} is deployed for hosting, customizing, and invoking the language model. The Ollama gls{api} facilitates seamless interaction between the client application and the server-hosted language model.

\gls{json}\footnote{\url{https://www.json.org/}} serves as the primary data exchange format between the server and client components. This lightweight and human-readable format is used both for constructing gls{api} calls to the Ollama gls{api} and for transmitting the language model's responses back to the client.
\begin{figure}[h]
    \centering
    \captionsetup{justification=centering}
    \includegraphics[width=0.9\textwidth]{graphics/techstack.png}
    \caption{Technology stack visualized on base architecture}
    \label{fig:techstack}
\end{figure}

\section{Server}
This section covers the server-side implementation of the Smart Home Chatbot system. 
It details the initial plan to use bwCloud and the challenges that led to switching to a private computer setup. 
We discuss the hardware specifications of the server and how it affects model performance and response times. 
The section also explains the deployment of Ollama v0.1.47 for hosting and customizing language models. 
We then dive into the process of model selection, considering factors like parameter count, popularity, and language support. 
Finally, we explore the techniques used for model customization, including the creation of modelfiles and prompt engineering, to tailor the language models for smart home interactions.

\subsection{Hardware and Performance}
\label{subsec:hardware}
The initial plan was to use the bwCloud\footnote{\url{https://www.bw-cloud.org/}}, a currently free service that can be used of students and researchers of different institutions accross Baden-Württemberg, Germany.
It was easy to get the needed ressources for this project which where eigth VCPUs, 16GB RAM and also enough memory for the size of the \glspl{llm} that were planned to use.
When trying out to run models directly on the server the response times were okay for models up to approximately 10 Billion perameters although the used server has no GPU.
However, when testing customized models through HTTP Requests the response times were much higher than expected.
Especially the first response often took over one minute with preceeding requests taking minimum 30 seconds depending on the length of the generated respone.
The first request usually takes longer when the model used is not allready loaded into the RAM.

Because of the occuring difficulties and with wanting to use as low budget as possible we decided to use an existing private Computer with more Ressources and used IPv6 Host Exposure for a specific port on which the Ollama gls{api} was running.
The computer had an NVIDIA GeForce 980 ti graphics card with 6GB VRAM and 32GB RAM with 3600MHz.
Even longer model responses usually only took a few seconds to receive.

\subsection{Model Customization}
\label{subsec:modelcust}
This section is all about the language models themselves. It covers which model where selected and why, how the models where customized with so called ``Modelfiles'' and an engineered prompt.
\subsubsection{Model Selection}
Initially the model in the focus of this work was llama3 since it was one of the most recent models and seen everywhere when starting with the .
However, it makes sense to try out other models and see how they perform therfore we came up with a solid model selection.
The model selection was based on parameter count, popularity, ranking in the \gls{bfcl} and availability in the ollama models library.
Another condition for the model was to support German.
The parameter count should be lower than 15 Billion since it wouldn't really run on the server setup described in the last section.
The model should be either under the most popular models filter on the Ollama library website \footnote{\url{https://ollama.com/library?sort=popular}}

% Define a new column type 'Y' for wider last column
\newcolumntype{Y}{>{\hsize=1.2\hsize}X}
\begin{table}[h!]
    \centering
    \begin{tabularx}{\textwidth}{lXXp{3cm}Xl}
    \toprule
    Model & Parameters & Size & Popularity \newline (Ollama) & BFCL \newline Accuracy & Organization \\
    \midrule
    home-3b-v3 & 3b & 1.7GB & 2.3K & - & - \\
    qwen2-7b-instruct & 7b & 4.4GB & 281.6K & - & Alibaba \\
    qwen2-1.5b-instruct & 1.5b & 0.9GB & 281.6K & - & Alibaba \\
    mistral-7b-instruct & 7b & 4.1GB & 2.8M & - & Mistral AI \\
    gemma-2b-instruct & 2b & 1.7GB & 3.9M & - & Google \\
    gemma-instruct & 7b & 5GB & 3.9M & 43.82 & Google \\
    zephyr & 7b & 4.1GB & 107.8K & - & Mistral AI \\
    llama3 & 8b & 4.7GB & 4.5M & - & Meta \\
    llama3-instruct & 8b & 4.7GB & 4.5M & 60.29 & Meta \\
    phi2 & 2.7b & 1.6GB & 200.1K & - & Microsoft \\
    phi3-3.8b & 3.8b & 2.2GB & 2.1M & - & Microsoft \\
    phi3-14b & 14b & 7.9GB & 2.1M & - & Microsoft \\
    gemma2 & 9b & 5.4GB & 290.1K & - & Google \\
    gorilla-openfunctions-v2 & 6.9b & 2.7GB & <1K & 84.65 & Gorilla LLM \\
    \bottomrule
    \end{tabularx}
    \caption{Overview of the models used.}
\end{table}

Here are some additional notes to this table:
\begin{itemize}
    \item The model \textit{home-3b-v3} was selected because it was fine-tuned to control devices with Home Assistant \cite{acon96_home_llm}.
    \item Only the commercial models from \textit{Mistral AI} are on the BCFL.
    \item \textit{Gemma2} and \textit{Phi3} were released later in the thesis phase, so \textit{gemma} and \textit{phi2} were also used.
    \item We noticed that the \textit{instruct} version of \textit{llama3} performed slightly better for our task than the plain version. Therefore, we usually preferred the instruct versions of other models if available.
    \item The model \textit{gorilla-openfunctions-v2} was not added to Ollama by Gorilla LLM but by a user who made it executable on Ollama.
    \item The \gls{bfcl} accuracy rating is according to the following date: 2024-07-06
    \item The popularity in the Ollama library is based on the amount of pulls of a model. It is one count per model which includes each different version of the model (e.g. instruct, different sizes).
\end{itemize}

\subsubsection{Modelfiles}
Modelfiles are configuration files used to customize and fine-tune language models for specific applications. They allow developers to define instructions, examples, and parameters that guide the model's behavior, ensuring more accurate and contextually appropriate responses. In the context of this project, modelfiles were crucial for tailoring the language model to understand and interact with the Bosch Smart Home system effectively.
The Ollama software uses modelfiles to create and share models. Let's examine the key components of our custom modelfile for the Bosch Smart Home chatbot:

\begin{Listing}
\begin{lstlisting}[language=bash]
FROM llama3:instruct
PARAMETER temperature 0.95
\end{lstlisting}
\caption{Base Model and Temperature Setting}
\label{lst:base_model}
\end{Listing}

The section in \cref{lst:base_model} specifies the base model (llama3:instruct) and sets the temperature parameter. As shown in this listing, the temperature value of 0.95 allows for more creative responses while maintaining coherence.

\begin{Listing}
\begin{lstlisting}[language=bash]
SYSTEM """
You are 'SHBot' (Smart Home Bot), a helpful AI Assistant that controls smart home devices. Complete tasks or answer questions based on a provided device list. Always respond in the language of the user request and keep answers brief.
Answer in the following format containing a natural language response to the user and a json:
natural language answer to the user
{
'action': 'intent/action',
'value': 'optional value for an action',
'deviceID': 'ID of the device',
'device': 'device type',
'room': 'device room',
'name': 'device name'
}
Important: Always place the json at the end of your response.
"""
\end{lstlisting}
\caption{System Message - Part 1: Role Definition and Response Format}
\label{lst:system_message_1}
\end{Listing}

Listing \ref{lst:system_message_1} shows the first part of the system message, which defines the bot's role and sets the expected response format. It instructs the model to provide both a natural language answer and a structured \gls{json} output, which is crucial for integrating the chatbot's responses with the smart home system.

\begin{Listing}
\begin{lstlisting}[language=bash]
SYSTEM """
Available Actions:
'none', 'turn-on', 'turn-off', 'change-temperature(value needed)'
Devices:
'socket': can use 'turn-on' and 'turn-off'
'thermostat' or 'RADIATOR_THERMOSTAT': The measured temperature can be viewed on each individual thermostat (RADIATOR_THERMOSTAT). It is typically structured like this: "id": "TemperatureLevel", "state": { "temperature": 22.5 }
'room-climate-control': can use 'change-temperature(value)'. This is a virtual device in the smart home that manages the temperature (called setPointTemperature) of the thermostats in the same room. If no thermostat exists, the system won't create a room climate control.
'door-window-contact': can't use the actions, only provides information whether its opened or closed.
"""
\end{lstlisting}
\caption{System Message - Part 2: Available Actions and Device Types}
\label{lst:system_message_2}
\end{Listing}

Listing \ref{lst:system_message_2} outlines the available actions and device types in the Bosch Smart Home system. This section provides the model with crucial information about the capabilities and limitations of each device type, enabling more accurate responses to user queries.

\begin{Listing}
\begin{lstlisting}[language=bash]
SYSTEM """
Example conversations:
For this example assume the user has the following devices:
'[{"type": "POWER_METER_SWITCH", "name": "TV", "deviceID": "device123", "state": [{"id": "PowerSwitch", "state": {"switchState": "OFF"}}], "room": "Schlafzimmer"}, {"type": "SHUTTER_CONTACT", "name": "WindowSensor-67890", "deviceID": "device456", "state": [{"id": "ShutterContact", "state": {"value": "CLOSED"}}], "room": "Wohnzimmer"}]'

MESSAGE user Can you turn on my TV?
MESSAGE assistant Sure, turning on the TV now. { "action": "turn-on", "deviceID": "device123", "device": "POWER_METER_SWITCH", "room": "Schlafzimmer", "name": "TV" }
MESSAGE user Sind alle Fenster geschlossen?
MESSAGE assistant Ja, alle Fenster sind geschlossen.
"""
\end{lstlisting}
\caption{Example Conversations and Device List}
\label{lst:example_conversations}
\end{Listing}

Listing \ref{lst:example_conversations} provides example conversations to guide the model's responses. It includes a sample device list and demonstrates how the model should interpret user queries and format its responses, including the use of different languages and the structured \gls{json} output.

\paragraph{Using Example User Inputs for the Modelfiles} 
A significant aspect of our model customization process involved leveraging the example user inputs collected during our survey mentioned in \cref{sec:collectinputs}. We carefully selected a subset of these inputs to include in the modelfile used for customizing the selected \glspl{llm}. This approach allowed us to fine-tune the models with real-world, domain-specific examples, enhancing their ability to understand and respond to typical smart home queries. By incorporating these authentic user inputs into the model's training data, we aimed to improve its contextual understanding and response accuracy for smart home-related interactions and therefore create a more domain-specfic model.

By incorporating these detailed instructions and examples, as shown in Listings \ref{lst:base_model} through \ref{lst:example_conversations}, the modelfile ensures that the language model can effectively understand and respond to user queries about their Bosch Smart Home devices, providing accurate information and executing commands as needed.


\subsubsection{Prompt Engineering}

Prompt engineering is the process of designing and refining input prompts to elicit desired responses from language models. It involves crafting specific instructions, context, and examples to guide the model's output effectively. In the context of our Bosch Smart Home chatbot, prompt engineering was crucial for ensuring accurate and relevant responses to user queries about their smart home devices.

The process of building the prompt is closely intertwined with the development of the modelfile. In our modelfile, we defined how the model should act and specified that it would be provided with a device list for each user. This integration of prompt engineering and modelfile development was essential for creating a cohesive and effective chatbot system.

Several approaches were considered for implementing the prompt engineering:

\begin{enumerate}
    \item \textbf{Utilizing context data:} Some recent models can handle additional context data alongside the user prompt. This would allow for the device list and other relevant information to be provided through this mechanism.

    \item \textbf{Dynamic SYSTEM message updates:} This approach involves changing the SYSTEM message dynamically with each user's device list.

    \item \textbf{Incorporating the device list in the user message:}
    \begin{enumerate}
        \item Building a message history where the first message always contains the current device list, followed by the user's actual message.
        \item Combining the device list and user message in a single prompt, formatted as ``devices: $<>$, message:$<>$''.
        \item Extending the previous approach with a ``mode'' parameter, alternating between ``get-data'' and ``answer-user'' modes based on whether the model has sufficient information to respond.
    \end{enumerate}
\end{enumerate}

Due to our objective of testing different models, the first option of using context data was not suitable, as it would limit our flexibility in model selection.

The ``mode'' approach showed promise in initial tests but proved complex to implement fully. Determining when to switch between modes and managing background calls to the model when it needed more data presented significant challenges.

Dynamically changing the SYSTEM message for each interaction was considered inefficient due to the overhead of updating and sending the entire message via gls{api} for every request. We briefly considered implementing a server-side function to update the modelfile with the device list, but ultimately chose a different method.

The final approach we adopted was building a message history. This method proved to be intuitive and straightforward to implement on the client side. By including the device list as the first message in the history, followed by the user's actual query, we could provide the necessary context to the model without overly complicating the implementation or limiting our ability to test different models.

This approach allowed us to maintain flexibility in our model selection while effectively providing the necessary context for accurate responses to user queries about their Bosch Smart Home devices.


\section{Client}
This section provides details about all the components at the client-side as approached in \cref{sec:design}. 
The client-side implementation is a crucial part of the Smart Home Chatbot system, handling user interactions, data management, and communication with the server. It extends the existing Bosch Smart Home Android app, integrating new functionalities while maintaining consistency with the app's design and user experience. 
The following subsections break down the various aspects of the client implementation, including how data is managed, the user interface design, message handling, request construction, and response processing. 
Each component plays a vital role in creating a seamless and efficient chatbot experience for smart home users.

\subsection{Data Management}
Additionally to the \gls{json} data exchange between client and server the data of the smart home itself has to be managed in order to put it into the \gls{json} and transmit it and also handle the \gls{json} received from the server.
The Bosch Smart Home app has an internal data model to store the current state of the users smart home. It updates this model when starting the app and on certain events.
Therefore, two options were available for managing the device data: either using the available data model or sending requests through the available internal gls{api} of the system.
To keep the latency low we deviced to use the internal data model and filter relevant data out. As described in \cref{subsec:devices} three devices were considered for the prototype.
Therefore we only used data relevant for the user from this devices. Since there was much unnecessary data in each device state we collected only the data described in \cref{tab:device_state_data_detailed}
\begin{table}[h!]
    \centering
    \begin{tabularx}{\textwidth}{|p{3cm}|X|}
    \hline
    \textbf{Device Type} & \textbf{Collected State Data} \\ \hline
    Smart Plug & 
    \textbf{powerConsumption}: Current power usage in watts \newline
    \textbf{energyConsumption}: Total energy consumed over time in kilowatt-hours \newline
    \textbf{switchState}: Whether the device is turned ON or OFF \\ \hline
    Thermostat & 
    \textbf{valvePosition}: Position of the radiator valve \newline
    \textbf{childLock}: Whether the child lock is ON or OFF \newline
    \textbf{temperature}: Current temperature measured by the thermostat in Celsius \\ \hline
    Room Climate Control & 
    This is not a real device. It is virtually managing all thermostats in one room to achieve a desired temperature.
    \textbf{ventilationMode}: Whether ventilation mode is active. \newline
    \textbf{boostMode}: Whether boost mode is active. This mode is used to increase the heat output for a short time \newline
    \textbf{operationMode}: Current operation mode (automatic or manual) \newline
    \textbf{setpointTemperature}: Desired temperature set by the user in Celsius \newline
    \textbf{currentTemperature}: Actual room temperature in Celsius \\ \hline
    Door Window Contact & 
    \textbf{contactState}: Whether the window or door is OPEN or CLOSED \\ \hline
    \end{tabularx}
    \caption{Detailed state data collected from different smart home devices}
    \label{tab:device_state_data_detailed}
\end{table}

Of course when controling devices the data model has to be updated in parallel to updating the devices themselves.

Another point for data management are the system logs of the Bosch Smart Home system to for example reason why something happened in the system and data from external sources to for example analyze how the electricity price changed.
Since we were not able to implement this it is mentioned in \cref{sec:challenges-solutions}.


\subsection{User Interface}
\label{sec:ui}
% in the main activity in the Bosch Smart Home app multiple views can be selected in the bottom like "Favourites" and "Room". The Toolbar in the top is always shown while in the main activity. Extending this existing functionality we added a button saying "CHATBOT" to the toolbar which can be seen in \cref{fig:ui-homescreen} to have the chatbot functionality available no matter where you are currently in the main activity. This provides a simple but efficient way to get to the chatbot.
% In \cref{fig:chatactivity} an overview of the actual chatbot \gls{ui} is shown. The toolbar here consists of an "Up Button" how it is often called in Android which is just a button to get back to the main activity. It also has an heading "Smart Home Chatbot".
% Directly under the toolbar another bar is shown which is more of an development feature. It makes it possible to select different models (that can also correspond to different endpoints) through a dropdown (spinner) and is shown more detailed in \cref{fig:model-select}. A message is always sent to the currently selected model/endpoint.
% Just below this model selection the heart of the chatbot \gls{ui} can be seen which is the chat history consisting of the messages the user wrote at the right in an olive green and the messages of ``SHBot'' (abbreviation of Smart Home Chatbot) on the left with white background. The time on which a message was sent/received is also shown.
% The message history shown in \cref{fig:ui-chatactivity} shows a conversation about the functionality of the chatbot. As you can see it summarized its functionality quite well by describing possible action, providing an example and also limitations. Only a bit buggy is the last sentence which is probably due to the fact that the chatbot usually answers with natural language in the beginning and a \gls{json} in the end of a message.
% As a note: the chat history is only persisted in the current session and is lost upon reentering the chatbot activity which was sufficient for the prototypical use.
% In the bottom of the whole chatbot \gls{ui} the input field for the user and a button to send his message are available. When clicking on the field the keyboard of the Android device expands.
% In our case the keyboard also has feature to use speech-to-text.

The user interface of our Smart Home Chatbot was designed to seamlessly integrate with the existing Bosch Smart Home app while providing easy access to the chatbot functionality. Figure \ref{fig:ui-overview} provides an overview of the user interface implementation.
The chatbot activity is built upon the Android AppCompatActivity class, ensuring compatibility across different Android versions and providing a consistent look and feel with the rest of the application.
In the main activity of the Bosch Smart Home app, users can select multiple views such as ``Favourites'' and ``Rooms'' using the bottom navigation. To make the chatbot easily accessible from any part of the main activity, we extended the existing functionality by adding a ``CHATBOT'' button to the toolbar, as shown in Figure \ref{fig:ui-homescreen}. This simple yet efficient solution allows users to access the chatbot from anywhere within the main activity.

Figure \ref{fig:ui-chatactivity} displays the layout of the chatbot \gls{ui}. The toolbar features an ``Up Button'', a common Android navigation element, allowing users to return to the main activity. The toolbar also includes the heading  ``Smart Home Chatbot'' for clear identification.
Below the toolbar, we implemented a development feature that allows selection of different models or endpoints through a dropdown menu (spinner), as detailed in Figure \ref{fig:model-select}. This feature enables testing and comparison of various models during development. Messages are always sent to the currently selected model/endpoint.

\begin{figure}[t]
    \centering
      \begin{subfigure}[t]{.44\textwidth}
        \vspace*{0pt}
        \includegraphics[width=\textwidth]{graphics/homescreen.jpg}
        \caption{Availability of the chatbot in the main app activity}
        \label{fig:ui-homescreen}
      \end{subfigure} \hfill
      \begin{subfigure}[t]{.44\textwidth}
        \vspace*{0pt}
        \includegraphics[width=\textwidth]{graphics/chatactivity.jpg}
        \caption{View of the chatbot activity}
        \label{fig:ui-chatactivity}
        \end{subfigure}
      \caption{Overview of the User Interface}
      \label{fig:ui-overview}
\end{figure}

The core of the chatbot \gls{ui} is the chat history, displayed below the model selection. User messages appear on the right side in olive green, while responses from ``SHBot'' (Smart Home Chatbot) are shown on the left with a white background. Each message includes a timestamp indicating when it was sent or received.
A notable feature of the chatbot's response mechanism is the real-time display of each token as it is received from the language model. This creates the effect of the chatbot ``writing'' its response in real-time, enhancing the interactive feel of the conversation and providing immediate feedback to the user.
The conversation shown in Figure \ref{fig:ui-chatactivity} demonstrates the chatbot's ability to summarize its functionality, describing possible actions, providing examples, and outlining limitations. A minor inconsistency is noted in the last sentence, likely due to the chatbot's typical response format of natural language followed by \gls{json} data.
It's worth noting that the chat history is only maintained for the current session and is not persisted when re-entering the chatbot activity, which was deemed sufficient for the prototype.

\begin{figure}[b]
    \centering
      \begin{subfigure}[t]{.48\textwidth}
        \vspace*{0pt}
        \includegraphics[width=\textwidth]{graphics/model-select.jpg}
        \caption{Expanded Model Selection}
        \label{fig:model-select}
      \end{subfigure} \hfill
      \begin{subfigure}[t]{.43\textwidth}
        \vspace*{0pt}
        \includegraphics[width=\textwidth]{graphics/keyboard.jpg}
        \caption{View when writing a message}
        \label{fig:keyboard}
        \end{subfigure}
      \caption{Details of the Chatbot User Interface}
      \label{fig:ui-details}
\end{figure}

At the bottom of the chatbot \gls{ui}, users can find an input field for typing messages and a send button. When the input field is selected, the Android device's keyboard appears, as shown in Figure \ref{fig:keyboard}. The keyboard also includes a speech-to-text feature, enhancing accessibility and user convenience.
Figure \ref{fig:ui-details} provides additional details of the chatbot user interface, showcasing the expanded model selection dropdown and the view when composing a message.

This user interface design, built on AppCompatActivity, ensures that the Smart Home Chatbot is easily accessible, intuitive to use, and seamlessly integrated with the existing Bosch Smart Home app functionality. The real-time token display feature adds a dynamic and engaging element to the user experience.

\subsection{Message Management}
As previously described in \cref{subsec:messageadapter}, the Message Adapter module manages and triggers the displaying of chat messages in the gls{ui}, making it possible to render message data, managing chat history, visually differentiating user and assistant messages, and enabling dynamic updates without full gls{ui} refreshes. \\
The implementation of this module leverages Android's RecyclerView component, which provides an efficient and flexible way to display large sets of data. RecyclerView is particularly well-suited for chat applications due to its ability to recycle and reuse view holders, minimizing memory usage and enhancing scrolling performance.


The Message Adapter extends RecyclerView.Adapter and implements a custom ViewHolder pattern. This pattern allows for efficient view recycling and type-specific rendering of messages. Two main types of ViewHolders are defined: one for user messages and another for assistant responses. This differentiation enables distinct visual styling for each message type as shown in , enhancing readability and user experience.\\
To manage the chat history, the adapter maintains an internal list of message objects. Each message object encapsulates data such as the message content, timestamp, and sender type (user or assistant). The adapter provides methods to add new messages and update existing ones, triggering appropriate gls{ui} updates through notifyItemInserted() and notifyItemChanged() methods respectively.

Dynamic updates are achieved through the use of DiffUtil, an Android utility class that calculates the difference between two lists. When new messages are added or existing ones are updated (even thought updating mussages is not supported in our prototype), DiffUtil computes the minimal set of changes needed to update the gls{ui}, allowing for smooth animations and efficient rendering.
To ensure a responsive user interface, message loading and processing operations are performed asynchronously using Java's ExecutorService. This approach prevents blocking the main thread from tasks like waiting for building the request to the chatbot and awaiting its answer.


\subsection{Request Building}
\label{sec:req-building}
As previously described in \cref{subsec:apiclient}, the Request Builder within the gls{api} Client \& Request Builder module constructs properly formatted gls{api} requests, sends them, parses server responses, and handles asynchronous operations.
This section elaborates on the process of building and sending these requests.

To ensure low latency while maintaining context, the system considers only a limited number of recent messages when constructing a request. Importantly, the first message in the history always contains a list of the user's devices, providing crucial context for the language model.
The message history follows a strict alternating pattern between ``user'' and ``assistant'' roles after the initial device list. When a user sends a new message, it is appended to this structured history, ensuring the language model has sufficient context for accurate reasoning.
\cref{lst:post-request} illustrates the format of a POST request sent to the server. The 'messages' array encapsulates the condensed message history, with the user's latest message positioned at the end. This structure provides the language model with a concise yet comprehensive context for generating appropriate responses.

\captionsetup[lstlisting]{labelformat=empty}
\begin{Listing}[h]
\begin{minipage}{0.53\textwidth}
    \begin{lstlisting}[caption={Base Structure of each POST Request}, label=lst:first, frame=single]
POST http://<domain-placeholder>/api/chat
{
    "model": "<model-placeholder>",
    "messages": [
    {
        "role": "user",
        "content": "<device-list-placeholder>"
    },
    {
        "role": "user",
        "content": "<user-message-placeholder>"
    }
    ],
    "stream": true
}
    \end{lstlisting}
    \end{minipage}
    \hfill
    \begin{minipage}{0.4\textwidth}
    \vspace{10pt}
    \begin{lstlisting}[caption={Example Device List containing only a Smart Plug}, label=lst:second, frame=single]
[{
    "type": "POWER_METER_SWITCH",
    "name": "Office Desk Lamp",
    "deviceID": "12345",
    "state": [
        {
        "id": "PowerMeter",
        "state": {
            "powerConsumption": 10,
            "energyConsumption": 50
        }
        },
        {
        "id": "PowerSwitch",
        "state": {
            "switchState": "ON"
        }
        }
    ],
    "room": "Office"
}]
    \end{lstlisting}
    \end{minipage}
    \caption{Components of a POST Request to the Server Running Ollama}
    \label{lst:post-request}
\end{Listing}
\captionsetup{labelformat=default}

The 'model' field specifies the language model to be used, while the 'stream' parameter set to true enables real-time streaming of the model's response. This approach allows for immediate display of partial responses, enhancing the user experience by reducing perceived latency.
The device list, exemplified in the right panel of \cref{lst:post-request}, provides detailed information about each smart home device, including its type, name, unique identifier, current state, and location. This comprehensive device context enables the language model to generate informed and relevant responses to user queries about their smart home environment.


\subsection{Response Handling and Action Triggering}
As previously described in \cref{subsec:responsehandler}, the Response Handler module interprets server responses, determining necessary client actions such as response analysis and orchestration. The response handling is closely integrated with the gls{api} Client, which parses the server responses and initiates the response handling process. This section elaborates on the implementation details of this crucial functionality.
The response from the server, as initiated by the request detailed in \cref{sec:req-building}, is received as a stream of tokens through the Ollama software. This streaming approach allows for real-time processing and display of the model's output, enhancing the responsiveness of the chatbot interface.
While receiving the response stream, the handler separates it into two components:

Natural Language Response: This portion is immediately forwarded to the gls{ui} for display, maintaining a fluid conversation flow with the user.\\
\gls{json} Object: Typically positioned at the end of the response, this structured data undergoes parsing to extract action-related information.

The \gls{json} object, as defined in the modelfile (see \cref{lst:system_message_1}), contains key information for action triggering (only listed the important ones here):

\begin{itemize}
\item 'action': Specifies the intent or action to be performed
\item 'value': An optional parameter for actions requiring additional data
\item 'deviceID': Unique identifier of the target device
\end{itemize}

The Response Handler processes this \gls{json} object to determine the appropriate action. As outlined in \cref{lst:system_message_2}, the system supports several actions:

\begin{itemize}
\item 'none': No action required
\item 'turn-on': Activate a device (e.g., a smart socket)
\item 'turn-off': Deactivate a device
\item 'change-temperature': Adjust temperature settings (requires a 'value' parameter)
\end{itemize}

Based on the 'action', 'deviceID' and eventually the 'value' fields in the \gls{json}, the Response Handler triggers the corresponding function in the client application with correct prarameters. For instance:

\begin{itemize}
\item If the action is 'turn-on' or 'turn-off', it calls the appropriate method to change the state of the specified device (identified by 'deviceID').
\item For 'change-temperature', it invokes the temperature adjustment function for the room climate control device, using the provided 'value'.
\item In case of 'none', no further action is taken beyond displaying the natural language response.
\end{itemize}

The main challenge here was to design a robust parsing mechanism for mapping the received \gls{json} to the exact fucntionality wanted.
The Response Handler also manages error scenarios, such as invalid actions or device IDs not present in the current device list. In such cases, it generates appropriate error messages for the user and logs the issues for system maintenance.

\section{Interaction Flow}
The interaction flow between the user, client-side application, and server is illustrated in Figure \ref{fig:sequencedia}. This sequence diagram outlines the process of handling user queries, constructing gls{api} requests, and generating responses based on smart home device information.

The flow begins with user input via the chatbot interface. The client-side application then retrieves the current state of smart home devices and packages this information, along with the user's query and recent message history, into a structured gls{api} request using the Request Builder module.
This request is sent to the server hosting the Ollama software and language model, which processes the input and generates a response. The response is streamed back to the client in real-time, enhancing user experience by reducing perceived latency.

The Response Handler module separates the incoming response into a natural language component and a structured gls{json} object. The natural language portion is displayed in the gls{ui}, while the gls{json} object is parsed for any action directives.

If actions are specified, the Response Handler triggers the corresponding functions in the client application, potentially sending commands to smart home devices or updating the application's state. The gls{ui} is then updated to reflect both the chatbot's response and any changes in device states.
This finnishes the interaction loop. The entire process typically occurs within seconds.

\begin{figure}[h]
    \centering
    \captionsetup{justification=centering}
    \includegraphics[width=\textwidth]{graphics/sequencedia.png}
    \caption{Sequence Diagram of the typical flow of the system}
    \label{fig:sequencedia}
\end{figure}

\section{Challenges and Solutions}
\label{sec:challenges-solutions}
There were several challenges and considerations that had to be taken into account when designing and developing the prototype:

\begin{itemize}
    \item \textbf{Complexity of Automations}: Balancing simplicity and functionality in user-defined automations proved challenging. The vast possibilities of Bosch Smart Home automations made it difficult to build a comprehensive assistant. Even if the language model could understand the capabilities of the automations, mapping this to a \gls{json} output and triggering a function to create specific automations would be complex. A potential solution would have been to define a set of automation types that could be created, but time constraints prevented this development.
    
    \item \textbf{Multilingual Support}: Providing accurate and contextually appropriate responses in both German and English was difficult in some scenarios. The chatbot occasionally struggled to consistently answer in German when the user's last input was in German, despite understanding the request. A possible reason could be that the device list provided to the model is always in English. Potential solutions include providing the model with information about the Bosch Smart Home app's current language, adapting the device list's language accordingly, or fine-tuning the model. Time limitations prevented the implementation of these solutions.

    \item \textbf{Historical Data}: During the thesis, historical data of the Bosch Smart Home devices was not available. Only the current state of devices could be accessed. Some variables in this state store summarized historical data, such as the total power consumption for smart plugs. A potential solution could have been to mock the data to test what data format would be sufficient for the model to correctly answer user requests and interpret the data.

    \item \textbf{Availability of System Logs}: Beginning with the second iteration of intents, system logs would have acted as a solid foundation for short-term historical data, including state changes of devices, automations, and errors. However, this information is only available inside the Bosch Smart Home Controller and therefore encapsulated from the Android App. Time constraints prevented the development of functionality to share these logs with the app.

    \item \textbf{Data from External Sources}: Integrating data from external sources, such as the ``Statistisches Bundesamt'' (Federal Statistical Office of Germany), could have provided valuable insights for energy cost analysis in the user's home. This data, including information on inflation in energy costs (gas, district heating, electricity, etc.), could have been used to analyze why energy costs increased and offer suggestions for energy savings based on unnecessary automations. Implementing this would require data retrieval and ingestion from defined sources (URLs), potentially using \gls{rag} with Ollama or Langchain. While time constraints prevented the implementation of this feature in our prototype, we explored the necessary data and methods. Our experiments showed that providing the language model with a list of devices containing (artificial) historical data, information on related automations, and inflation indices gave sufficient context for reasoning, although not always with perfect accuracy.
    
    \item \textbf{Security Issue}: The Bosch Smart Home app initially did not support gls{api} calls to arbitrary domains. To address this, we needed to modify the security settings for the prototype to enable calls to the Ollama gls{api} on a specific device. This change was implemented only for the chatbot functionality, while the rest of the app maintained its original security configuration.

    \item \textbf{Model Performance}: Incrementally improving model outputs proved challenging. In our case, this was achieved by analyzing the output of our evaluation script, identifying low-quality outputs, and addressing them in the modelfile by adding more examples or instructive text. The evaluation script itself is described in \cref{sec:modelperform}.
\end{itemize}
%\blinddocument

% !TeX spellcheck = en_US
\lstset{
    basicstyle=\ttfamily\small,
    breaklines=true,
    numbers=left,
    numberstyle=\tiny,
    stepnumber=1,
    numbersep=5pt,
    %backgroundcolor=\color{gray!10},
    frame=lr,
    %captionpos=b,
    tabsize=2,
    keepspaces=true,
    showspaces=false,
    showstringspaces=false,
    showtabs=false,
    keywordstyle=\bfseries,
    commentstyle=\itshape\color{gray},
    stringstyle=\ttfamily\color{darkgray},
    lineskip=0.1em  % Add space between lines
}

\chapter{Evaluation}
\label{chap:evaluation}

In this chapter, we present the evaluation methodology and results for the smart home chatbot. The evaluation focuses on two key aspects: the semantic similarity of the responses and the accuracy of the generated JSON commands. Additionally, we discuss the initial approach using classification metrics, the challenges encountered, and the refined approach to address these challenges.

\section{Study Design}
In this section we want to provide the base study design we came up.
The design of our evaluation was gathered through clearly defining the goals of the evaluation and coming to measurable metrics in the end via the top-down \gls{gqm} approach.
Our approach consists of three main goals: assessing the accuracy, the user experience and the explainability of the developed smart home chatbot.
Based on this we developed the whole evaluation process which consists of a \gls{llm} evaluation approach for the accuracy and a user study for the other two goals.
Details are provided later within this chapter.

\subsection{Goal Question Metric Paradigm}
The \gls{gqm} paradigm according to \citet{caldiera1994goal} provides a structured approach to evaluate different works in the area of Software Engineering and therefore is also suitable for evaluating various aspects of the smart home chatbot. 
Our evaluation framework consists of three primary goals, each addressing a specific area of interest: accuracy, user experience, and explainability.
This framework is shown in \cref{fig:gqm}

\textbf{Goal 1: Assess the Accuracy of the Smart Home Chatbot}

The first goal focuses on determining how accurately the chatbot can understand and respond to user commands. To achieve this, several questions are formulated:

\begin{itemize}
    \item \textbf{Q1: How accurate are the natural language answers of the language model?}
    \item \textbf{Q2: How accurate are the JSON responses of the language model?}
\end{itemize}

To answer these questions, relevant metrics are identified. Semantic similarity measures are used to evaluate the natural language responses, potentially incorporating other related metrics to ensure comprehensive assessment. JSON accuracy metrics are employed to evaluate the precision of the chatbot's structured responses. A combined metric of semantic similarity and JSON accuracy provides a holistic view of the chatbot's overall accuracy.

\textbf{Goal 2: Evaluate the User Experience of the Smart Home Chatbot}

The second goal is to understand the users' interaction experience with the chatbot. This involves evaluating how intuitive and satisfactory the chatbot is in performing tasks. The questions under this goal include:

\begin{itemize}
    \item \textbf{Q1: Are typical tasks easy to achieve?}
    \item \textbf{Q2: How satisfied are users with the chatbot's performance?}
    \item \textbf{Q3: What could be improved?}
    \item \textbf{Q4: Does the chatbot add to existing functionality of typical smart home applications?}
\end{itemize}

The metrics for these questions involve measuring task completion time, the number of attempts, and the success rate of task completion. User satisfaction is gauged through questionnaires administered after the experiment. These questionnaires assess various aspects of the user experience, including ease of use, overall satisfaction, and areas for improvement.

\textbf{Goal 3: Assess the Explainability of the Smart Home Chatbot}

The third goal addresses how well the chatbot can explain its actions and decisions to users, which is crucial for building trust and usability. The questions related to this goal are:

\begin{itemize}
    \item \textbf{Q1: How clear and understandable are the chatbot's explanations?}
    \item \textbf{Q2: What could be improved?}
\end{itemize}

To measure the explainability, semi-structured interviews are conducted after the experiment. These interviews delve into the clarity, transparency, and usefulness of the explanations provided by the chatbot, allowing for detailed qualitative feedback from users.

\begin{figure}[h]
    \centering
    \captionsetup{justification=centering}
    \includegraphics[width=0.98\textwidth]{graphics/gqm.png}
    \caption{Visualized Goal Question Metric}
    \label{fig:gqm}
\end{figure}
\begin{figure}[h]
    \centering
    \captionsetup{justification=centering}
    \includegraphics[width=0.75\textwidth]{graphics/eval-process.png}
    \caption{The whole evaluation process visualized}
    \label{fig:evalprocess}
\end{figure}

\subsection{Resulting Evaluation Process}
A Visualization of our evaluation process can be seen in \cref{fig:evalprocess}
Based on the obtained \gls{gqm}, the evaluation can be split into two parts: evaluating the model performance and a user study for examining User Experience and Explainability.
Besides the developed protoype chatbot the obtained sample user inputs can be greatly used in the evaluation process.
They could be used to construct the dataset that was essential for the evaluation of the language model.
This evaluation dataset contains for each sample input an expected natural language output and an eventually expected \gls{json} to measure both the accuracy of the output the user sees and the constructed \gls{json} that is used for further actions in the smart home system.

The other part is the user study in which users have a setup of devices that are supported by our prototype and receive a list of tasks in which the success should be measured and afterwards a questionnaire and a semi-structured interview are used to answer the questions regarding Goal 2 and 3 in the defined \gls{gqm}

Based on these two parts, the takeaway of this thesis can be received and constructed.
The results emerge directly as an output from the model evaluation and the user study when combined with quantitative and qualitative methods.

Based on this and the detailed evaluation setup the results can be discussed and threats to validity be debated.


\section{Model Performance}
\label{sec:modelperform}


\subsection{Hardware and Language Models used}
The hardware setup was the same as described in \cref{subsec:modelcust} since it is very fast for the model sizes we wanted to test as also explained in the same section.
For measuring the model performance there was no need to have a setup of smart home devices since when the outputed \gls{json} of the model is correct the correct action will be triggered in the underlying system.
The language models we selected for evaluation are also described in \cref{subsec:modelcust}.

\subsection{Evaluation Dataset}
To assess the performance and capabilities of our smart home chatbot, we developed a comprehensive evaluation dataset with a total of 80 entries. This dataset is designed to simulate realistic user interactions and test the chatbot's ability to understand context, control devices, and provide informative responses.
The evaluation dataset consists of a series of input-output pairs, where each input represents a chat history and the output represents the expected response from the chatbot. 
In creating the inputs for our evaluation dataset, we leveraged the user inputs collected during our earlier survey (as described in Section 4.4). These real-world examples provided authentic phrasing and diverse query formulations typical of smart home users. We adapted and expanded upon these collected inputs to ensure they aligned with our specific test scenarios and device setups. This approach allowed us to create a more realistic and challenging evaluation dataset, closely mimicking the variety of natural language inputs a smart home chatbot would encounter in practical use.
The structure of each entry in the dataset is as follows:

\begin{enumerate}
    \item Input: A chat history containing a minimum of two messages from the user. The first message always includes a device list that provides crucial context about the user's smart home environment. For details on the device list structure, refer to \cref{sec:req-building}.
    \item Expected Output: A natural language response that the chatbot is expected to generate based on the given chat history.
    \item Expected JSON: A JSON object representing the action the chatbot should take, if any. The JSON includes only the necessary keys for each action:
    \begin{itemize}
    \item For 'turn-on' or 'turn-off' actions: 'action' and 'deviceID'
    \item For 'change-temperature' action: 'action', 'deviceID', and 'value'
    \item If no action is necessary, the expected JSON is "None"
    \end{itemize}
    \end{enumerate}

\begin{table}[htb]
    \centering
    \makebox[\textwidth]{\includegraphics[width=1.2\textwidth]{graphics/evaldata.pdf}}
    \caption{Format and Example Entries of the Evaluation Dataset}
    \label{tab:dataset-format}
\end{table}
    
Based on the chat history a request can be build with a python script leveraging langchain to easily access Ollama since it is supported by langchain.
That way we do not have to construct API calls ourself only parse the message history into a langchain function call that does everything in the background.
We can just create an ChatOllama object like the following:

\begin{lstlisting}[numbers=none, frame=none]
# Initialize language model
llm = ChatOllama(
    base_url="http://127.0.0.1:5000",
    model="sh-llama3-instruct",
    keep_alive=-1
)
\end{lstlisting}

Therefore we can easily switch between all models that we want to test out. The keep alive option set to minus one means that the model is loaded undefinetly into memory.
After parsing the messages of one csv entry we can just use the following code to create the code and invoke the language model:

\begin{lstlisting}[numbers=none, frame=none]
prompt = ChatPromptTemplate.from_messages(messages)
result = invoke_language_model(llm, prompt)
\end{lstlisting}

To create a diverse and representative dataset, we used 10 example device lists as the basis for our scenarios. These lists were carefully crafted to cover various smart home setups:

\begin{itemize}
    \item 5 edge case device lists, including:
    \begin{itemize}
    \item Thermostats in different rooms
    \item Multiple thermostats in the same room
    \item Multiple smart plugs with similar names in the same room
    \item Multiple door/window contacts in the same room
    \item A setup where the user has no thermostats
    \end{itemize}
    \item 3 random German examples with device and room names in German
    \item 2 random English examples with typical device names and variations of supported devices
\end{itemize}

These device lists were sometimes modified (e.g., changing variable values) to match specific test cases, ensuring a wide range of scenarios for evaluation.
The dataset covers various interaction types, including device control commands, queries about device states, requests for information about the smart home setup, and complex questions requiring reasoning about multiple devices or rooms.
Table \ref{tab:dataset-format} illustrates the format of the dataset and provides two example entries (note that the device lists are shortened to one device and removed state information here for a better overview, the actual device lists contain 2-6 devices).

This carefully created dataset allows us to evaluate the chatbot's performance across multiple dimensions, including accuracy in interpreting user intent, ability to provide relevant responses, correct identification and execution of required actions, contextual understanding, and handling of edge cases and ambiguous requests.


\subsection{Evaluation Metrics}
In this section we describe the metrics we used to measure the accuracy of different language models that we customized with our modelfile as described in \cref{subsec:modelcust}.
For this we have more deeply analyzed the metrics stated in the related work \cref{sec:relatedeval}.
We employed a combination of metrics that evaluate both the semantic accuracy of the natural language responses and the correctness of the generated JSON commands to reliably capture the relevant dimensions of our chatbot prototype.

\subsubsection{Semantic Similarity}

Semantic similarity is a crucial metric in our evaluation since it measures how closely the generated responses from the chatbot match the expected outputs in terms of meaning. For this evaluation, we used the \texttt{SentenceTransformer} model, specifically the \texttt{paraphrase-MiniLM-L6-v2} variant, to compute cosine similarity between the embeddings of the generated responses and the expected outputs. A high similarity score indicates that the chatbot's response is semantically close to the expected answer, even if the exact wording differs.

The process of calculating semantic similarity involves the following steps: \\
Encoding: Both the reference (expected) outputs and the generated responses are encoded into high-dimensional vectors using the SentenceTransformer model.\\
Similarity Computation: The cosine similarity between the embeddings of the reference and generated responses is calculated. Cosine similarity measures the cosine of the angle between two vectors, providing a value between -1 and 1, where 1 indicates perfect similarity, 0 indicates no similarity, and -1 indicates perfect dissimilarity.\\
Aggregation: The individual similarity scores are aggregated to produce an average similarity score across all samples in the evaluation dataset.\\

\begin{Listing}[htb]
    \begin{lstlisting}[language=Python]
def calculate_semantic_similarity(references, generated_responses):
    model = SentenceTransformer('paraphrase-MiniLM-L6-v2')
    embeddings1 = model.encode(references, convert_to_tensor=True)
    embeddings2 = model.encode(generated_responses, convert_to_tensor=True)
    cosine_scores = util.pytorch_cos_sim(embeddings1, embeddings2)

    similarities = [cosine_scores[i][i].item() for i in range(len(references))]
    average_similarity = sum(similarities) / len(similarities)
    return similarities, average_similarity
  \end{lstlisting}
    \caption{Code for calculating the semantic similarity through cosine similarity}
    \label{lst:similarity}
\end{Listing}

The implementation of this metric is shown in \cref{lst:similarity}. This code defines a function \texttt{calculate\_semantic\_similarity} that takes two lists of sentences (references and generated responses) as input and returns a list of individual similarity scores along with the average similarity score.
A high average similarity score indicates that the chatbot's responses are semantically close to the expected answers, even if the exact wording differs. This allows for a more flexible evaluation that captures the chatbot's ability to understand and respond to user intents accurately, rather than merely reproducing exact phrases.


\subsubsection{JSON Accuracy}
We define \gls{json} accuracy as the percent of correctly generated \glspl{json} by a language model customized to our use case.
Since our dataset contains only necessary keys of the expected \gls{json}, a correct generated \gls{json} is one which contains each expected key and the correct corresponding value for it.

Each generated \gls{json} is compared against the expected \gls{json} to determine if it correctly represents the intended action or response. 
The code in \cref{lst:evalMetrics1} shows how we have implemented this. The accuracy we just explained is represented by the variable ``accuracy''.

The total count (total\_count) is the total number of generated responses, which represents the total number of \glspl{json} evaluated. This is determined by the length of the generated\_responses list.
The correct\_count is increased in several scenarios:
\begin{enumerate}
    \item When both the expected and generated \glspl{json} are None.
    \item When the expected \gls{json} is None and the generated \gls{json} has an "action" key with the value "none".
    \item When the generated \gls{json} matches the expected \gls{json} in terms of keys and their corresponding values (determined by the compare\_jsons function).
\end{enumerate}
Generated \glspl{json} that would throw an error on parsing are handled as incorrect. This is implemented in the code through the use of try-except blocks. If a JSONDecodeError occurs when trying to parse the expected \gls{json}, or if an AttributeError occurs when comparing the generated and expected \glspl{json}, the \gls{json} is considered incorrect and the json\_accuracy\_flags for that instance is set to False.
The accuracy is then calculated by dividing the correct\_count by the total\_count.

The code also includes the variable ``key\_accuracy'' which checks how many keys have the correct value as in the expected \gls{json}. It is calculated as follows:
\begin{itemize}
    \item total\_keys is incremented for each key in the expected \gls{json}.
    \item correct\_keys is incremented when a key in the generated \gls{json} matches the corresponding key in the expected \gls{json}.
    \item The key\_accuracy is then calculated as the ratio of correct\_keys to total\_keys.
\end{itemize}
This key\_accuracy provides a more granular measure of how well the generated \glspl{json} match the expected \glspl{json} on a key-by-key basis, even if the entire \gls{json} doesn't match perfectly.

The function shown returns three values: the overall \gls{json} accuracy, the key accuracy, and a list of boolean flags indicating which generated \glspl{json} were correct (json\_accuracy\_flags).

\begin{Listing}
    \begin{lstlisting}[language=Python]
def evaluate_jsons(generated_responses, generated_jsons, expected_json_values):
    correct_count = total_keys = correct_keys = 0
    total_count = len(generated_responses)
    json_accuracy_flags = []

    for response, generated_json, expected_json in zip(generated_responses, generated_jsons, expected_json_values):
        if expected_json is not None and isinstance(expected_json, str):
            try:
                expected_json = json.loads(expected_json)
            except json.JSONDecodeError:
                json_accuracy_flags.append(False)
                continue        
        if expected_json is None and generated_json is None:
            correct_count += 1
            json_accuracy_flags.append(True)
            continue
        if expected_json is None:
            if generated_json.get("action") == "none": 
                correct_count += 1
                json_accuracy_flags.append(True)
                continue
            json_accuracy_flags.append(False)
            continue        
        if generated_json is None:
            json_accuracy_flags.append(False)
            continue
        try:
            keys_correct = compare_jsons(generated_json, expected_json)
            if keys_correct:
                correct_count += 1
                json_accuracy_flags.append(True)
            else:
                json_accuracy_flags.append(False)
            
            for key in expected_json:
                total_keys += 1
                if generated_json.get(key) == expected_json.get(key):
                    correct_keys += 1
        except AttributeError:
            json_accuracy_flags.append(False)
    
    accuracy = correct_count / total_count
    key_accuracy = correct_keys / total_keys if total_keys > 0 else 0
    return accuracy, key_accuracy, json_accuracy_flags
  \end{lstlisting}
    \caption{Code for Classificiation of the models responded JSONs}
    \label{lst:evalMetrics1}
\end{Listing}

\begin{Listing}
    \begin{lstlisting}[language=Python]
def normalize_value(value):
    """Normalize the value for comparison."""
    try:
        # Try to convert strings that represent numbers to float
        return float(value)
    except (ValueError, TypeError):
        # If it's not a number or it's already a number, return it as is
        return value

def compare_jsons(generated_json, expected_json):
    """Compare two JSON objects with normalized values."""
    if generated_json is None or expected_json is None:
        return generated_json == expected_json
    
    for key in expected_json:
        if key not in generated_json:
            return False
        # normalize value if the key is "value"
        if key == "value":
            return normalize_value(generated_json[key]) == normalize_value(expected_json[key])
        else:
            return generated_json[key] == expected_json[key]
    return True
    \end{lstlisting}
    \caption{Code for comparing actual and expected JSONs}
    \label{lst:compare-json}   
\end{Listing}


\subsection{Combined Metric}

To provide a holistic view of the chatbot's performance, we developed a combined metric that integrates both semantic similarity and \gls{json} accuracy. This approach was inspired by the need to evaluate the chatbot's performance across multiple dimensions simultaneously.
The combined metric adapts the concepts of precision, recall, and F1 score from traditional classification tasks to our specific use case. Here's how we define the components:

\gls{tp}: Cases where the model's output has high semantic similarity (above a defined threshold) and the generated \gls{json} is correct.\\
\gls{fp}: Cases where the model's output has high semantic similarity but the generated \gls{json} is incorrect.\\
\gls{tn}: Cases where the model's output has low semantic similarity and the  generated \gls{json} is incorrect.\\
\gls{fn}: Cases where the model's output has low semantic similarity but the generated \gls{json} is correct.

Based on these definitions, we calculate precision, recall, and F1 score as follows:

\paragraph{Precision:} Of all the outputs the model that have a high semantic similarity, how many were actually positive cases (generated \gls{json} is also correct).
\paragraph{Recall:} Of all the actual positive cases (generated \gls{json} is correct), how many did the model correctly identify as positive (semantic similarity is also high).
\paragraph{F1 Score:} The harmonic mean of precision and recall, providing a single measure of performance.

The implementation of this combined metric is shown in \cref{lst:classificationRefined} and based on the scikit-learn library\footnote{\url{https://scikit-learn.org/stable/modules/model\_evaluation.html\#precision-recall-and-f-measures}}. This function, \texttt{calculate\_classification\_metrics}, takes lists of similarity scores and \gls{json} accuracy flags as input, along with a similarity threshold. It then computes the precision, recall, and F1 score based on our adapted definitions.
It's important to note that this approach, while not standard for non-classification tasks, provides valuable insights into our chatbot's performance. By combining semantic similarity and \gls{json} accuracy, we can evaluate how well the model meets both criteria simultaneously, which is crucial for its functionality in a smart home system.
The similarity threshold is a critical parameter that determines what constitutes "high" semantic similarity. This threshold should be chosen based on domain knowledge and experimentation to ensure that the similarity measure is robust and meaningful for the specific use case.
By using this combined metric, we can quantify the chatbot's ability to provide semantically appropriate responses while also generating accurate \gls{json} commands. This approach aligns well with the practical expectations of the chatbot's performance in a real-world smart home setting.

\begin{Listing}
    \begin{lstlisting}[language=Python]
    def calculate_classification_metrics(similarities, json_accuracy_flags, similarity_threshold=0.8):
    y_true = []
    y_pred = []

    for similarity, json_correct in zip(similarities, json_accuracy_flags):
        # True label is positive if JSON is correct
        y_true.append(1 if json_correct else 0)

        # Predicted positive if similarity is above threshold and JSON is correct
        if similarity >= similarity_threshold and json_correct:
            y_pred.append(1)
        else:
            y_pred.append(0)

    # Calculate precision, recall, and F1 score
    precision = precision_score(y_true, y_pred)
    recall = recall_score(y_true, y_pred)
    f1 = f1_score(y_true, y_pred)

    return precision, recall, f1
  \end{lstlisting}
    \caption{Refined Classification Metrics}
    \label{lst:classificationRefined}
\end{Listing}

\subsubsection{Determining the Similarity Threshold}
The choice of an appropriate similarity threshold is crucial for the effectiveness of our combined metric. Through careful analysis of the semantic similarity between various generated outputs and their corresponding expected outputs, we determined that a threshold of 0.65 provides the most meaningful differentiation. This decision was based on several key observations:

\textbf{Cross-language Differentiation}: We found that when a generated response was in English but the expected response was in German, the similarity score was consistently below 0.55. This served as a clear demarcation point, even when considering the constraints of the language model size we chose.

\textbf{Quality Assessment}: After establishing 0.55 as a baseline, we examined cases with higher similarity scores. Our analysis revealed that responses with a similarity score greater than 0.65 were consistently of high quality and relevance.

\textbf{Borderline Cases}: We identified examples that fell just below and above our chosen threshold to validate its appropriateness. For instance, an unacceptable response with a similarity of approximately 0.61 was: "I cannot have a temperature measurement for the kitchen. There are only rooms and devices, but no temperature-taking devices available for the kitchen." The expected response was: "I'm sorry, but no temperature sensors are available in the kitchen." Despite some semantic overlap, this response was deemed insufficient.
It is worth noting that this is a literal translation from a German example, i.e., when calculating the similarity between those two sentences it will probably differ from the actual one.

\textbf{Acceptable Responses}: Conversely, we found that responses just above the 0.65 threshold were generally acceptable. An example with a similarity of about 0.67 was: "The temperature set point is 18 degrees, and the current temperature is 19 degrees." The expected response was: "The temperature in the Schlafzimmer is set to 18.0 degrees Celsius and the current temperature is 19.0 degrees Celsius." This response, while missing some specifics, captured the essential information accurately.

By setting the threshold at 0.65, we strike to conduct a strict and reliable evaluation. This threshold effectively distinguishes between responses that capture the core meaning and intent of the expected output and those that fall short, providing a robust basis for our combined metric.



\section{User Experience}

To evaluate the user experience of our smart home chatbot, we designed and conducted a comprehensive user study. This study aimed to assess the usability, effectiveness, and potential benefits of integrating a chatbot interface into a smart home system.

\subsection{Participants}
We recruited a diverse group of 10 participants for our study, comprising three employees from Bosch Smart Home and seven external participants, including university students and employees from other companies. The participants represented a varied demographic in terms of age and professional background, with 60\% having a technical background and 40\% non-technical. Experience with smart home systems varied among participants, with only one having no prior experience. Familiarity with chatbots was evenly distributed across the spectrum from no experience to regular use.

This diverse group was chosen to represent a range of potential users, from those familiar with smart home technology to novices, ensuring a comprehensive evaluation of the chatbot's accessibility and usefulness.

\subsection{Apparatus}
The experimental setup consisted of three smart home devices: a smart plug named "TV" and a thermostat, both located in the sleeping room, and a door/window contact in the living room. The language model ran on a PC with specifications described in Section [reference to hardware specs]. Participants interacted with the Bosch smart home app either through a PC emulating Android or a Samsung Galaxy S21 Smartphone. To ensure consistency, all participants used the same setup in the same room.

\subsection{Procedure}
Participants were asked to complete six tasks using the chatbot interface: inquiring about the chatbot's purpose, determining the state and energy consumption of the TV, checking temperature settings in the bedroom, requesting a summary of available devices, changing the bedroom temperature, and switching the TV's state. To mitigate learning effects, the order of these tasks was randomized for each participant.

\subsection{Metrics and Data Collection}
We collected both quantitative and qualitative data to evaluate the user experience. Quantitative metrics included task completion time, number of attempts per task, and task success rate. These were collected through automated logging and researcher observation. Qualitative data was gathered through a questionnaire using a 5-point Likert scale to assess various aspects of the chatbot's usability and effectiveness, followed by a semi-structured interview to explore participants' experiences in more depth.

\subsection{Analysis}
The data analysis plan incorporated both quantitative and qualitative approaches. Quantitative analysis focused on task completion rates, time taken, and survey scores. Qualitative analysis of interview responses aimed to identify common themes and areas for improvement. This mixed-methods approach allows for a comprehensive evaluation of the chatbot's performance and user perception, providing both statistical measures of effectiveness and rich, contextual insights into the user experience.


\section{Results}
... what is the result of your e.g., Questionnaire or experimentation.. 
Presentation of Findings
Data Analysis

% hypothesises: 1. controlling devices tasks generally take longer than tasks were no action needs to be triggered, 2. persons with technical background need less time for the tasks, 3. persons without technical background see more potential in the chatbot 4. Older persons are more concerned about the chatbot 5. The chatbot improves explainabiltiy of the app 6. The chatbot improves usability of the app
%
%
\subsection{Language Model Performance Results}
Using the refined approach, we obtained the following results:
\begin{itemize}
    \item \textbf{Precision}: [Value]
    \item \textbf{Recall}: [Value]
    \item \textbf{F1 Score}: [Value]
    \item \textbf{Semantic Similarity}: [Average Similarity]
    \item \textbf{JSON Accuracy}: [Accuracy Value]
\end{itemize}

\subsection{Qualitative Analysis Results}
The semi-structured interviews yielded rich qualitative data, which we analyzed using thematic analysis. Four main themes emerged from the participants' responses to the questions about their experience with the smart home chatbot:
\subsubsection{Chatbot Limitations and Frustrations}
Many participants reported frustrations with the chatbot's limitations. Common issues included the chatbot's inability to control devices as expected, misinterpretation of complex queries, and difficulty handling follow-up questions or maintaining appropriate context. One participant noted, "When [the chatbot] should control devices and it doesn't do it," while another mentioned, "Sometimes the chatbot misinterprets complex queries."
\subsubsection{User Interface and Interaction Preferences}
Participants expressed various preferences and suggestions for improving the chatbot's user interface and interaction methods. These included requests for example commands, voice input options, and more visual elements. One participant suggested, "Example of possible questions would be nice to show. A voice option would be nice. Maybe more style in the text bold etc." Another recommended "more visual elements like icons or graphics."
\subsubsection{Utility for Complex Tasks and Device Management}
Many participants found the chatbot particularly useful for complex tasks involving multiple devices or for obtaining quick overviews of their smart home system. One participant noted the chatbot would be "very helpful if I had many devices (50 windows, outlets), then questions like 'is any window open' [would be useful]." Another mentioned its utility "for complex automations and scenarios involving multiple devices."
\subsubsection{Interest in Advanced Features and Optimization}
Participants showed significant interest in advanced features, particularly those related to energy optimization and understanding system behavior. One participant expressed interest in "automatic suggestions/tips for optimizing behavior (usage patterns, power consumption...)." Another was "interested in understanding the logic behind my smart home's operations."

\subsection{User Experience Demonstration}
% include screenshots of example conversations 

\section{Discussion}
% not always response in correct language --> could be solved in telling the model in 
% which language to answer through the system language of the phone/app

% This work shows how a chatbot application for smart homes can be built. It could be transferred into a framwework where it would only be necessary to specify the domain to a self hosted language model or an commercial API with API key and the mapping functionality for parsing and mapping the output \gls{json} of the language model to the actions of the smart home system

... Based on the results argue about acceptance or rejection of your research hypothesis   .. 
Interpretation of Results
Comparison with Previous Studies
Limitations of the Study

\subsection{Interpretation of Qualitative Findings}
The qualitative data from our study provides valuable insights into users' experiences with the smart home chatbot and their expectations for such a system.
\subsubsection{Balancing Simplicity and Complexity}
Our findings reveal a tension between users' desire for simplicity and their interest in complex functionalities. While some participants found technical language challenging and requested simpler interactions, others were eager for advanced features like energy optimization and complex automations. This suggests that future iterations of the chatbot should offer tiered interaction levels, allowing users to choose between simplified and more advanced interfaces based on their comfort and expertise.
\subsubsection{Enhancing Contextual Understanding}
The frustrations expressed regarding the chatbot's contextual understanding and ability to handle complex queries highlight an area for significant improvement. Implementing more sophisticated natural language processing and context retention could greatly enhance the user experience. As one participant suggested, a feature to correct the chatbot's understanding mid-conversation could be particularly valuable.
\subsubsection{Leveraging Visual and Multi-modal Interactions}
The requests for more visual elements and voice input options indicate that users desire a more diverse and intuitive interaction experience. Incorporating these elements could make the chatbot more accessible and engaging for a wider range of users, potentially increasing its adoption and regular use.
\subsubsection{Focusing on High-Value Use Cases}
The strong interest in using the chatbot for complex tasks, device management, and system optimization suggests these are high-value use cases that should be prioritized in future development. The chatbot's ability to handle scenarios involving multiple devices and provide quick system overviews appears to be a significant advantage over traditional interfaces.
\subsubsection{Addressing Privacy and Scope Concerns}
While many participants were enthusiastic about advanced features, some expressed concerns about privacy and the appropriate scope of the chatbot's functionality. This highlights the need for transparent data handling practices and clear communication about the chatbot's capabilities and limitations.
\subsection{Implications for Future Development}
Based on these findings, future development of the smart home chatbot should focus on:
\begin{enumerate}
    \item Improving natural language processing to handle more complex and contextual queries.
    \item Developing a tiered interface that caters to both novice and advanced users.
    \item Incorporating more visual elements and possibly voice interaction capabilities.
    \item Enhancing capabilities for managing multiple devices and creating complex automations.
    \item Implementing features for energy optimization and system behavior insights.
    \item Ensuring robust privacy protections and clear communication about data usage.
\end{enumerate}


These enhancements could significantly improve user satisfaction and the overall utility of the chatbot in smart home environments.


\section{Threats to Validity}
% all participants had either interest in the product by working at Bosch Smart Home or be known by the researchers. Therefore positive resonance may be biased.
% mistakes in the evaluation datasets could occur since it was quite complex to build. - Lead to both: artificially better or worse results
%  classification through the work below?
... Discuss what threatens validity of your result. In case you could counteract them explain how. For experimentation in software engineering there is already a classification of this threats and a check-list \cite{DBLP:journals/ese/RunesonH09}.   
%LaTeX-Hinweise stehen in \cref{chap:latexhints}.

%noch etwas Fülltext
%\blinddocument

% !TeX spellcheck = en_US

\chapter{Conclusion}\label{chap:conclusion}
In this final chapter of the thesis we provide a brief summary of the most important apsect and insights. We also go over benefits of our results but also limitations where our work does not hold. 
Lastly we we mention lessons learned and additionally suggestions for future work.

\section{Summary}
This thesis explored the development and evaluation of an innovative chatbot system designed to enhance user interaction and system explainability in smart home environments. The research focused on integrating \glspl{llm} without fine-tuning to support diverse intents within the Bosch Smart Home ecosystem, addressing the growing complexity of smart home technologies.

A novel approach combining natural language processing with \gls{json} function calling was implemented, enabling the chatbot to handle both device control and data analysis tasks. The system architecture, utilizing a client-server model with Android Studio and Ollama, demonstrated practical deployment potential in real-world scenarios.

The evaluation process introduced a comprehensive metric that combined semantic similarity and \gls{json} accuracy, providing a nuanced assessment of the chatbot's performance. This revealed interesting trade-offs between \gls{json} accuracy and semantic understanding across different model iterations. 
The \texttt{shllama3instruct} model (as we named our customized model) emerged as the top performer, balancing high semantic similarity with good precision in \gls{json} function calling.

User studies validated the chatbot's positive impact on system explainability and usability. However, challenges remained in user preference for the chatbot over traditional interfaces and in optimizing device control task efficiency. Qualitative feedback highlighted the importance of balancing simplicity with advanced functionality and addressing privacy concerns in smart home applications.

The research uncovered potential for developing a generalized framework for smart home chatbots, which could significantly impact future developments in smart home technology. Areas for improvement were identified, including enhanced contextual understanding, integration of multimodal inputs or evenoutputs, better customization of the language model to always answer in the language of the last user message and the addressing of complex use cases, especially on the topic energy consumption.

This thesis contributes to the field by demonstrating the feasibility and benefits of integrating recent \glspl{llm} into smart home systems. It offers valuable insights into the challenges and opportunities in enhancing user interaction with complex smart home ecosystems, paving the way for more intuitive and explainable smart home technologies.

\section{Benefits}
% ... Who (software testers, software developers, software architects,etc.) benefits from your result? In what way?   ..
% I am missing the ``in what way'' for the most points here, please add
% software architects and software developers through identifying especially open source tools and building blocks for implementing a chatbot in the smart home domain
% people that want to customize models without fine tuning
% gls{llm} interested people / researches that are interested in the performance of models in the size range we tested
% Companies like Bosch Smart Home that are interested in which data is interesting for such a chatbot while at the same time have company central goals like preserving high customer data security/privacy. Our approach does not let customer data leave the company since the customized gls{llm} is running on an internal server.
The thesis results benefit various stakeholders in the smart home and software development domains. Software architects and developers gain insights into open-source tools and building blocks for implementing chatbots in smart home systems. Individuals interested in customizing models without fine-tuning can leverage the approach demonstrated. 
\gls{llm} researchers benefit from performance data on models within the tested size range, providing valuable benchmarks for future studies. 
Companies like Bosch Smart Home, focused on maintaining high customer data security and privacy, can utilize the approach that keeps customer data within the company by running customized \glspl{llm} on internal servers. 
This addresses the crucial balance between functionality and data protection. Additionally, the research offers insights into which data types are most relevant for smart home chatbots, helping companies prioritize data collection and usage. 
Overall, the thesis contributes to advancing smart home technology while addressing key industry concerns, making it valuable for both technical professionals and business stakeholders in the smart home sector.

\section{Limitations}
% In what settings your approach/method/theory does not hold/work?
% we only used open-source language models. commercial/Proprietary models may perform much different
% our function calling approach is only tried out with mapping JSONs to functionality although other modern ways of function calling are available but only for few models (e.g. Mistral). So we don't know how these other approaches would perform. Also, we don't know if our approach is suitable for other domains than smart home since our focus and data is specialices for this.
The study's limitations primarily stem from its focused scope and methodological choices. 
Firstly, the exclusive use of open-source language models means that the performance of commercial or proprietary models remains unexplored, potentially offering different results. 
The function calling approach, centered on mapping JSONs to functionality, doesn't account for newer methods available in select models like Mistral. This limitation raises questions about the approach's effectiveness in other domains beyond smart homes. 
Additionally, the specialization of data and focus on smart home applications may limit the generalizability of findings to other fields. 
Lastly, the study doesn't address the scalability of the approach for larger, more complex smart home systems with a wider range of devices and functionalities.

\section{Lessons Learned}
% ... What lessons did you learn throughout your thesis that is interesting to be mentioned? For example during experiment setup, or literature review, etc. ... 
% a multi-faceted evaluation approach like ours takes a lot of time. Even if it gives useful insights we would only focus on more certain aspects in a future work or split the work in two works were one focuses on the system archtitecture and model performance while a second work focuses on the user-centric side.
% Manual creation of the evaluation (or a training) dataset is also very time consuming. There exist
% There are tools for monitoring / regular evaluation of \gls{llm} models for Ollama. For example ollama-grid-search (https://github.com/dezoito/ollama-grid-search) or promptfoo (https://www.promptfoo.dev/). We didnt use such tools but they can speed up and improve the process. 
% The selection of evaluation metrics for use case of a developed \gls{llm} need to be carefully chosen. It is good to plan enough time for the selection and implementation of this.
The thesis process yielded valuable insights for future research endeavors. A key lesson was the time-intensive nature of multi-faceted evaluation approaches. 
While providing comprehensive insights, such methods demand significant resources. 
Future work might benefit from focusing on specific aspects or dividing the research into separate studies - one concentrating on system architecture and model performance, another on user-centric aspects. 
The manual creation of evaluation or training datasets proved exceptionally time-consuming, highlighting the need for more efficient data generation methods. 
The existence of tools for monitoring and regularly evaluating \gls{llm} models (for Ollama), such as ollama-grid-search\footnote{\url{https://github.com/dezoito/ollama-grid-search}} and promptfoo\footnote{\url{https://www.promptfoo.dev/}}, was noted as potentially beneficial for future studies to accelerate and enhance the process. 
Lastly, the importance of carefully selecting evaluation metrics for \gls{llm} use cases became evident. Allocating sufficient time for metric selection and implementation is crucial for ensuring the relevance and accuracy of research outcomes in this rapidly evolving field.

\section{Future Work}
% What is your suggestion for future work?
% Exploring larger parameter models or advanced fine-tuning/training approaches to enhance both JSON accuracy and semantic similarity.
% Developing more robust generalization capabilities to handle a wider range of user requests.
% Implementing a more readable format for device data presentation to the model to see if this affects the performance especially of other models than llama3
% For Bosch Smart Home especially further functionality that includes system logs and historical data would be benefital. It would be need to researched what a good aproach is to provide the system logs to a chatbot and how to filter relevant data out of it. Also for historical data it would be necessary to get insights into how the data has to look to answer specific questions. Another point here would be to investigate how and where to store this data.
% Investigat whether Incorporating more visual elements and voice interaction capabilities improve the user acceptance of such a chatbot
% Developing and evaluating features for energy optimization and system behavior insights.
% Conducting larger-scale user studies with more diverse participants and smart home setups.

There are various directions in which future research could go.
Exploring larger parameter models or advanced fine-tuning/training approaches could enhance both \gls{json} accuracy and semantic similarity. 
Developing more robust generalization capabilities would enable handling a wider range of intents and user requests.
Implementing a more readable format for device data presentation to the model could potentially improve performance, especially for models other than llama3. 

For Bosch Smart Home specifically, incorporating system logs and historical data functionality would be beneficial, necessitating research into effective data presentation and filtering methods for chatbots.
Developing and evaluating features for energy optimization and system behavior insights could add significant value to the Bosch Smart Home system.

Investigating the impact of visual elements and voice interaction capabilities on user acceptance is another area that could be researched. 
Further, conducting larger-scale user studies with diverse participants and smart home setups would provide more comprehensive insights. 

Finally, future research could also explore and categorize the various approaches to implementing \gls{ai} across different applications, including different training methodologies, customization techniques (such as optimizing system messages, parameters, and examples for zero-/one-/few-shot learning), and the feasibility of building custom models from scratch, to determine the most effective and efficient methods for specific use cases and domains.

\printbibliography

All links were last followed on July 14, 2024.

\appendix
% HINWEISE / TIPPS / HINTS
% \input{latexhints-english}

\pagestyle{empty}
\renewcommand*{\chapterpagestyle}{empty}
\Versicherung
\end{document}
