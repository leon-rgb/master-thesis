% !TeX spellcheck = en_US
\lstset{
    basicstyle=\ttfamily\small,
    breaklines=true,
    numbers=left,
    numberstyle=\tiny,
    stepnumber=1,
    numbersep=5pt,
    %backgroundcolor=\color{gray!10},
    frame=lr,
    %captionpos=b,
    tabsize=2,
    keepspaces=true,
    showspaces=false,
    showstringspaces=false,
    showtabs=false,
    keywordstyle=\bfseries,
    commentstyle=\itshape\color{gray},
    stringstyle=\ttfamily\color{darkgray},
    lineskip=0.1em  % Add space between lines
}

\chapter{Evaluation}
\label{chap:evaluation}

In this chapter, we present the evaluation methodology and results for the smart home chatbot. The evaluation focuses on two key aspects: the semantic similarity of the responses and the accuracy of the generated JSON commands. Additionally, we discuss the initial approach using classification metrics, the challenges encountered, and the refined approach to address these challenges.

\section{Study Design}
In this section we want to provide the base study design we came up.
The design of our evaluation was gathered through clearly defining the goals of the evaluation and coming to measurable metrics in the end via the top-down \gls{gqm} approach.
Our approach consists of three main goals: assessing the accuracy, the user experience and the explainability of the developed smart home chatbot.
Based on this we developed the whole evaluation process which consists of a \gls{llm} evaluation approach for the accuracy and a user study for the other two goals.
Details are provided later within this chapter.

\subsection{Goal Question Metric Paradigm}
The \gls{gqm} paradigm according to \citet{caldiera1994goal} provides a structured approach to evaluate different works in the area of Software Engineering and therefore is also suitable for evaluating various aspects of the smart home chatbot. 
Our evaluation framework consists of three primary goals, each addressing a specific area of interest: accuracy, user experience, and explainability.
This framework is shown in \cref{fig:gqm}

\textbf{Goal 1: Assess the Accuracy of the Smart Home Chatbot}

The first goal focuses on determining how accurately the chatbot can understand and respond to user commands. To achieve this, several questions are formulated:

\begin{itemize}
    \item \textbf{Q1: How accurate are the natural language answers of the language model?}
    \item \textbf{Q2: How accurate are the JSON responses of the language model?}
\end{itemize}

To answer these questions, relevant metrics are identified. Semantic similarity measures are used to evaluate the natural language responses, potentially incorporating other related metrics to ensure comprehensive assessment. JSON accuracy metrics are employed to evaluate the precision of the chatbot's structured responses. A combined metric of semantic similarity and JSON accuracy provides a holistic view of the chatbot's overall accuracy.

\textbf{Goal 2: Evaluate the User Experience of the Smart Home Chatbot}

The second goal is to understand the users' interaction experience with the chatbot. This involves evaluating how intuitive and satisfactory the chatbot is in performing tasks. The questions under this goal include:

\begin{itemize}
    \item \textbf{Q1: Are typical tasks easy to achieve?}
    \item \textbf{Q2: How satisfied are users with the chatbot's performance?}
    \item \textbf{Q3: What could be improved?}
    \item \textbf{Q4: Does the chatbot add to existing functionality of typical smart home applications?}
\end{itemize}

The metrics for these questions involve measuring task completion time, the number of attempts, and the success rate of task completion. User satisfaction is gauged through questionnaires administered after the experiment. These questionnaires assess various aspects of the user experience, including ease of use, overall satisfaction, and areas for improvement.

\textbf{Goal 3: Assess the Explainability of the Smart Home Chatbot}

The third goal addresses how well the chatbot can explain its actions and decisions to users, which is crucial for building trust and usability. The questions related to this goal are:

\begin{itemize}
    \item \textbf{Q1: How clear and understandable are the chatbot's explanations?}
    \item \textbf{Q2: What could be improved?}
\end{itemize}

To measure the explainability, semi-structured interviews are conducted after the experiment. These interviews delve into the clarity, transparency, and usefulness of the explanations provided by the chatbot, allowing for detailed qualitative feedback from users.

\begin{figure}[h]
    \centering
    \captionsetup{justification=centering}
    \includegraphics[width=\textwidth]{graphics/gqm.png}
    \caption{Visualized Goal Question Metric}
    \label{fig:gqm}
\end{figure}

\subsection{Resulting Evaluation Process}
A Visualization of our evaluation process can be seen in \cref{fig:evalprocess}
Based on the obtained \gls{gqm}, the evaluation can be split into two parts: evaluating the model performance and a user study for examining User Experience and Explainability.
Besides the developed protoype chatbot the obtained sample user inputs can be greatly used in the evaluation process.
They could be used to construct the dataset that was essential for the evaluation of the language model.
This evaluation dataset contains for each sample input an expected natural language output and an eventually expected \gls{json} to measure both the accuracy of the output the user sees and the constructed \gls{json} that is used for further actions in the smart home system.

The other part is the user study in which users have a setup of devices that are supported by our prototype and receive a list of tasks in which the success should be measured and afterwards a questionnaire and a semi-structured interview are used to answer the questions regarding Goal 2 and 3 in the defined \gls{gqm}

Based on these two parts, the takeaway of this thesis can be received and constructed.
The results emerge directly as an output from the model evaluation and the user study when combined with quantitative and qualitative methods.

Based on this and the detailed evaluation setup the results can be discussed and threats to validity be debated.


\begin{figure}[h]
    \centering
    \captionsetup{justification=centering}
    \includegraphics[width=\textwidth]{graphics/eval-process.png}
    \caption{The whole evaluation process visualized}
    \label{fig:evalprocess}
\end{figure}


\section{Model Performance}
\label{sec:modelperform}

... what was the settings of your designed study to evaluate your results? e.g., You designed a Questionnaire to assess if your method increases the productivity of a programmer, explain and justify what population you chose, what was the questions/tasks and all necessary details. If you designed an experiment against a software system to collect measures and assess accuracy of your model, i.e., the contribution of your research, here explain e.g., how you collected measurements, what was characteristics of machines, etc.


\subsection{Evaluation Dataset}
To assess the performance and capabilities of our smart home chatbot, we developed a comprehensive evaluation dataset. This dataset is designed to simulate realistic user interactions and test the chatbot's ability to understand context, control devices, and provide informative responses.
The evaluation dataset consists of a series of input-output pairs, where each input represents a chat history and the output represents the expected response from the chatbot. The structure of each entry in the dataset is as follows:

\begin{enumerate}
    \item Input: A chat history containing a minimum of two messages from the user. The first message always includes a device list that provides crucial context about the user's smart home environment. For details on the device list structure, refer to \cref{sec:req-building}.
    \item Expected Output: A natural language response that the chatbot is expected to generate based on the given chat history.
    \item Expected JSON: A JSON object representing the action the chatbot should take, if any. The JSON includes only the necessary keys for each action:
    \begin{itemize}
    \item For 'turn-on' or 'turn-off' actions: 'action' and 'deviceID'
    \item For 'change-temperature' action: 'action', 'deviceID', and 'value'
    \item If no action is necessary, the expected JSON is "None"
    \end{itemize}
    \end{enumerate}

To create a diverse and representative dataset, we used 10 example device lists as the basis for our scenarios. These lists were carefully crafted to cover various smart home setups:

\begin{itemize}
    \item 5 edge case device lists, including:
    \begin{itemize}
    \item Thermostats in different rooms
    \item Multiple thermostats in the same room
    \item Multiple smart plugs with similar names in the same room
    \item Multiple door/window contacts in the same room
    \item A setup where the user has no thermostats
    \end{itemize}
    \item 3 random German examples with device and room names in German
    \item 2 random English examples with typical device names and variations of supported devices
\end{itemize}

These device lists were sometimes modified (e.g., changing variable values) to match specific test cases, ensuring a wide range of scenarios for evaluation.
The dataset covers various interaction types, including device control commands, queries about device states, requests for information about the smart home setup, and complex questions requiring reasoning about multiple devices or rooms.
Table \ref{tab:dataset-format} illustrates the format of the dataset and provides two example entries (note that the device lists are shortened here for a better overview, the actual device lists contain 2-6 devices).

\lstset{
  basicstyle=\ttfamily\small,
  breaklines=true,
  postbreak=\mbox{\textcolor{red}{$\hookrightarrow$}\space},
}

\begin{longtable}{@{}p{0.45\textwidth}p{0.3\textwidth}p{0.2\textwidth}@{}}
    \toprule
    \textbf{Input} & \textbf{Expected Output} & \textbf{Expected JSON} \\
    \midrule
    \endhead
    
    \begin{lstlisting}[language=json]
    [
      {
        "role": "user",
        "content": "[
          {
            \"type\":\"POWER_METER_SWITCH\",
            \"name\":\"Zwischenstecker\",
            \"deviceID\":\"hdm:HomeMaticIP:3014F711A000049878593469\",
            \"state\":[
              {\"id\":\"PowerMeter\",\"state\":{\"powerConsumption\":0,\"energyConsumption\":0}},
              {\"id\":\"PowerSwitch\",\"state\":{\"switchState\":\"OFF\"}}
            ],
            \"room\":\"Schlafzimmer\"
          }
        ]"
      },
      {
        "role": "user",
        "content": "Turn on the Zwischenstecker, please."
      }
    ]
    \end{lstlisting}
    &
    The Zwischenstecker in your bedroom is now on.
    &
    \begin{lstlisting}[language=json]
    {
      "action": "turn-on",
      "deviceID": "hdm:HomeMaticIP:3014F711A000049878593469"
    }
    \end{lstlisting}
    \\
    \midrule
    
    \begin{lstlisting}[language=json]
    [
      {
        "role": "user",
        "content": "[
          {
            \"type\":\"POWER_METER_SWITCH\",
            \"name\":\"Wohnzimmerlampe\",
            \"deviceID\":\"hdm:HomeMaticIP:3014F711A000049878593472\",
            \"state\":[
              {\"id\":\"PowerMeter\",\"state\":{\"powerConsumption\":15,\"energyConsumption\":45}},
              {\"id\":\"PowerSwitch\",\"state\":{\"switchState\":\"ON\"}}
            ],
            \"room\":\"Wohnzimmer\"
          }
        ]"
      },
      {
        "role": "user",
        "content": "Can you turn on the Wohnzimmerlampe?"
      }
    ]
    \end{lstlisting}
    &
    The Wohnzimmerlampe is already turned on. No need to switch it on.
    &
    None
    \\
    
    \bottomrule
    \caption{Format and Example Entries of the Evaluation Dataset}
    \label{tab:dataset-format}
    \end{longtable}

This carefully curated dataset allows us to evaluate the chatbot's performance across multiple dimensions, including accuracy in interpreting user intent, ability to provide relevant responses, correct identification and execution of required actions, contextual understanding, and handling of edge cases and ambiguous requests.


\subsection{Evaluation Metrics}

\subsubsection{Semantic Similarity}

Semantic similarity measures how closely the generated responses from the chatbot match the expected outputs in terms of meaning. For this evaluation, we used the \texttt{SentenceTransformer} model, specifically the \texttt{paraphrase-MiniLM-L6-v2} variant, to compute cosine similarity between the embeddings of the generated responses and the expected outputs. A high similarity score indicates that the chatbot's response is semantically close to the expected answer, even if the exact wording differs.

\begin{Listing}
    \begin{lstlisting}[language=Python]
def calculate_semantic_similarity(references, generated_responses):
    model = SentenceTransformer('paraphrase-MiniLM-L6-v2')
    embeddings1 = model.encode(references, convert_to_tensor=True)
    embeddings2 = model.encode(generated_responses, convert_to_tensor=True)
    cosine_scores = util.pytorch_cos_sim(embeddings1, embeddings2)

    similarities = [cosine_scores[i][i].item() for i in range(len(references))]
    average_similarity = sum(similarities) / len(similarities)
    return similarities, average_similarity
  \end{lstlisting}
    \caption{Code for calculating the semantic similarity through cosine similarity}
    \label{lst:similarity}
\end{Listing}


\subsubsection{JSON Accuracy}

JSON accuracy evaluates the correctness of the structured data outputs generated by the chatbot. Each generated JSON is compared against the expected JSON to determine if it correctly represents the intended action or response. The accuracy is calculated by the proportion of correct JSONs to the total number of JSONs evaluated.

\begin{Listing}
    \begin{lstlisting}[language=Python]
def evaluate_jsons(generated_responses, generated_jsons, expected_json_values):
    correct_count = total_keys = correct_keys = 0
    total_count = len(generated_responses)
    json_accuracy_flags = []

    for response, generated_json, expected_json in zip(generated_responses, generated_jsons, expected_json_values):
        if expected_json is not None and isinstance(expected_json, str):
            try:
                expected_json = json.loads(expected_json)
            except json.JSONDecodeError:
                json_accuracy_flags.append(False)
                continue        
        if expected_json is None and generated_json is None:
            correct_count += 1
            json_accuracy_flags.append(True)
            continue
        if expected_json is None:
            if generated_json.get("action") == "none": 
                correct_count += 1
                json_accuracy_flags.append(True)
                continue
            json_accuracy_flags.append(False)
            continue        
        if generated_json is None:
            json_accuracy_flags.append(False)
            continue
        try:
            keys_correct = compare_jsons(generated_json, expected_json)
            if keys_correct:
                correct_count += 1
                json_accuracy_flags.append(True)
            else:
                json_accuracy_flags.append(False)
            
            for key in expected_json:
                total_keys += 1
                if generated_json.get(key) == expected_json.get(key):
                    correct_keys += 1
        except AttributeError:
            json_accuracy_flags.append(False)
    
    accuracy = correct_count / total_count
    key_accuracy = correct_keys / total_keys if total_keys > 0 else 0
    return accuracy, key_accuracy, json_accuracy_flags
  \end{lstlisting}
    \caption{Code for Classificiation of the models responded JSONs}
    \label{lst:evalMetrics1}
\end{Listing}

\begin{Listing}
    \begin{lstlisting}[language=Python]
def normalize_value(value):
    """Normalize the value for comparison."""
    try:
        # Try to convert strings that represent numbers to float
        return float(value)
    except (ValueError, TypeError):
        # If it's not a number or it's already a number, return it as is
        return value

def compare_jsons(generated_json, expected_json):
    """Compare two JSON objects with normalized values."""
    if generated_json is None or expected_json is None:
        return generated_json == expected_json
    
    for key in expected_json:
        if key not in generated_json:
            return False
        # normalize value if the key is "value"
        if key == "value":
            return normalize_value(generated_json[key]) == normalize_value(expected_json[key])
        else:
            return generated_json[key] == expected_json[key]
    return True
    \end{lstlisting}
    \caption{Code for comparing actual and expected JSONs}
    \label{lst:compare-json}   
\end{Listing}

\subsection{Initial Approach Using Classification Metrics}

\subsubsection{Description of the Approach}

Initially, we attempted to evaluate the chatbot using traditional classification metrics: precision, recall, and F1 score. In this context, we defined the true positives (TP), false positives (FP), true negatives (TN), and false negatives (FN) based on the correctness of the JSON outputs and the semantic similarity scores.

\begin{lstlisting}[language=Python, caption=Classification Metrics]

\end{lstlisting}

\begin{Listing}
    \begin{lstlisting}[language=Python]
def calculate_classification_metrics(similarities, json_accuracy_flags, similarity_threshold=0.8):
    y_true = []
    y_pred = []

    for similarity, json_correct in zip(similarities, json_accuracy_flags):
        y_true.append(1 if json_correct else 0)
        y_pred.append(1 if similarity >= similarity_threshold and json_correct else 0)

    precision = precision_score(y_true, y_pred)
    recall = recall_score(y_true, y_pred)
    f1 = f1_score(y_true, y_pred)

    return precision, recall, f1
  \end{lstlisting}
    \caption{Classification Metrics}
    \label{lst:evalMetrics2}
\end{Listing}

\subsubsection{Challenges Encountered}

The classification metrics approach presented several challenges:
\begin{itemize}
    \item \textbf{Misleading FP Cases}: The approach could not effectively capture false positives because when the JSON was correct, the predictions were always marked as true positive, thus leading to an absence of FP cases.
    \item \textbf{Definition Misalignment}: The standard definitions of TP, FP, TN, and FN did not perfectly align with our use case, where both semantic similarity and JSON correctness were critical but evaluated differently than in binary classification tasks.
\end{itemize}

\subsection{Refined Approach}

\subsubsection{Adjustments to the Evaluation Method}

To address the issues with the initial approach, we refined our evaluation method to better capture the nuances of our use case:
\begin{itemize}
    \item \textbf{Combined Metric for Positive Predictions}: A positive prediction is now defined as having both a semantic similarity score above 0.65 and a correct JSON output.
    \item \textbf{Refined Definitions}: We redefined TP, FP, TN, and FN to better suit our chatbot's evaluation context.
\end{itemize}

\begin{Listing}
    \begin{lstlisting}[language=Python]
    def calculate_classification_metrics(similarities, json_accuracy_flags, similarity_threshold=0.8):
    y_true = []
    y_pred = []

    for similarity, json_correct in zip(similarities, json_accuracy_flags):
        # True label is positive if JSON is correct
        y_true.append(1 if json_correct else 0)

        # Predicted positive if similarity is above threshold and JSON is correct
        if similarity >= similarity_threshold and json_correct:
            y_pred.append(1)
        else:
            y_pred.append(0)

    # Calculate precision, recall, and F1 score
    precision = precision_score(y_true, y_pred)
    recall = recall_score(y_true, y_pred)
    f1 = f1_score(y_true, y_pred)

    return precision, recall, f1
  \end{lstlisting}
    \caption{Refined Classification Metrics}
    \label{lst:classificationRefined}
\end{Listing}

\subsubsection{Evaluation Results}

Using the refined approach, we obtained the following results:
\begin{itemize}
    \item \textbf{Precision}: [Value]
    \item \textbf{Recall}: [Value]
    \item \textbf{F1 Score}: [Value]
    \item \textbf{Semantic Similarity}: [Average Similarity]
    \item \textbf{JSON Accuracy}: [Accuracy Value]
\end{itemize}

These results indicate that the refined evaluation method provides a more accurate and reliable assessment of the chatbot's performance in handling user queries and controlling smart home devices.

\section{User Experience}

... what was the settings of your designed study to evaluate your results? e.g., You designed a Questionnaire to assess if your method increases the productivity of a programmer, explain and justify what population you chose, what was the questions/tasks and all necessary details. If you designed an experiment against a software system to collect measures and assess accuracy of your model, i.e., the contribution of your research, here explain e.g., how you collected measurements, what was characteristics of machines, etc.

% tasks were shuffled to compensate learning effects
% 


\section{Results}
... what is the result of your e.g., Questionnaire or experimentation.. 
Presentation of Findings
Data Analysis
\subsection{User Experience Demonstration}
% include screenshots of example conversations 

\section{Discussion}
% not always response in correct language --> could be solved in telling the model in 
% which language to answer through the system language of the phone/app

% This work shows how a chatbot application for smart homes can be built. It could be transferred into a framwework where it would only be necessary to specify the domain to a self hosted language model or an commercial API with API key and the mapping functionality for parsing and mapping the output \gls{json} of the language model to the actions of the smart home system

... Based on the results argue about acceptance or rejection of your research hypothesis   .. 
Interpretation of Results
Comparison with Previous Studies
Limitations of the Study


\section{Threats to Validity}
% all participants had either interest in the product by working at Bosch Smart Home or be known by the researchers. Therefore positive resonance may be biased.
% mistakes in the evaluation datasets could lead to worse results

... Discuss what threatens validity of your result. In case you could counteract them explain how. For experimentation in software engineering there is already a classification of this threats and a check-list \cite{DBLP:journals/ese/RunesonH09}.   
%LaTeX-Hinweise stehen in \cref{chap:latexhints}.

%noch etwas Fülltext
%\blinddocument
