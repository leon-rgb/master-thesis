% !TeX spellcheck = en_US

\chapter{Evaluation}
\label{chap:evaluation}

Here, you need to discuss about evaluation of your research result. You may wanna take a look at Goal Question Metric (GQM) paradigm that helps you in the process of evaluation \cite{caldiera1994goal}

\section{Study Design}
... what was the settings of your designed study to evaluate your results? e.g., You designed a Questionnaire to assess if your method increases the productivity of a programmer, explain and justify what population you chose, what was the questions/tasks and all necessary details. If you designed an experiment against a software system to collect measures and assess accuracy of your model, i.e., the contribution of your research, here explain e.g., how you collected measurements, what was characteristics of machines, etc.

\section{Results}
... what is the result of your e.g., Questionnaire or experimentation.. 
Presentation of Findings
Data Analysis

\section{Discussion}
... Based on the results argue about acceptance or rejection of your research hypothesis   .. 
Interpretation of Results
Comparison with Previous Studies
Limitations of the Study

\section{Threats to Validity}
... Discuss what threatens validity of your result. In case you could counteract them explain how. For experimentation in software engineering there is already a classification of this threats and a check-list \cite{DBLP:journals/ese/RunesonH09}.   
%LaTeX-Hinweise stehen in \cref{chap:latexhints}.

%noch etwas Fülltext
%\blinddocument
