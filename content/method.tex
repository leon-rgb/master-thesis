\chapter{Methodology}
\label{chap:method}
This chapter outlines the research methodology employed in this thesis, focusing on the scientific methods used to address the research objectives.

The research methodology of this master's thesis is a comprehensive, mixed-methods approach that incorporates elements of design science and exploratory and empirical research.
This approach is well-suited for a thesis in the field of smart home technologies which are deeply connected with human-computer interaction while we at the same time also take a look at the architecture and building blocks of the intended chatbot.
Due to its structured nature, design science enables an iterative development of the chatbot and the solving of specific problems and therefore is commonly used in software engineering research.
On the other hand the explorative empirical approach aids in understanding user requirements and gathering real-world data to inform the design process and in the end collecting and analyzing evidence that is observable and measurable.

\paragraph{Methods Used:}
%Literature Review including the topics: classification and types of chatbots, typical building blocks used in chatbots, function calling in language models and similar chatbot research, for example smart home chatbots for simple tasks or chatbots for complex scenarios like using multiple knowledge bases, evaluation techniques used for similar language model based chatbots
%iterative
%The chatbot was developed iteratively, as visualized in \cref{fig:iterative-design} of the Introduction chapter. Each iteration involved planning, implementation, testing, and evaluation phases to refine the chatbot's functionality based on performance. Initially, the approach was to add new intents with each iteration. However, the focus shifted to exploring how to develop the first intents effectively and iteratively improving their addressing. This adjustment allowed for a more thorough understanding and enhancement of the initial functionalities.
%preliminary interviews with experts of different domains to gather requirements and ideas for the chatbot development. This qualitative data informed the initial design and development stages.
%survey for gathering example user inputs for the chatbot. Resulting from the preliminary interviews we conducted a survey to gather example user inputs for the chatbots which were essential for developing and also the evaluation
%prototype implementation
%A prototype of the chatbot was designed and implemented within the Bosch Smart Home system, utilizing a client-server architecture. The technology stack included open-source tools to support the development and integration processes.
%study design:
%model performance mostly quantitative but also qualitative analysis of what did work out what did not and why, including the building of an evaluation dataset partly based on the  
%quantitative and qualitative analysis through an experiment: user study with tasks to complete with the chatbot, questionnaire and semi-structured interviews.

A comprehensive literature review was conducted, covering several key areas relevant to chatbot development and smart home technologies. This review explored the classification and types of chatbots, typical building blocks used in their construction, and the emerging technique of function calling in language models. Additionally, similar chatbot research was examined, including studies on smart home chatbots for simple tasks and more complex scenarios involving multiple knowledge bases. The review also investigated evaluation techniques used for language model-based chatbots, providing a solid foundation for the evaluation methodology employed in this study.

The development process followed an iterative approach, as illustrated in \cref{fig:iterative-design} of the Introduction chapter. Initially, the strategy was to introduce new intents with each iteration. However, as the project progressed, the focus shifted towards developing the initial intents more effectively and iteratively improving their implementation. This adjustment enabled a more thorough understanding and enhancement of the core functionalities.

To gather requirements and ideas for the chatbot development, preliminary interviews were conducted with experts from various domains. This qualitative data was instrumental in informing the initial design and development stages, ensuring that the chatbot's features aligned with real-world needs and expectations.

Following the insights gained from the preliminary interviews, a survey was conducted to collect example user inputs for the chatbot. These user inputs were crucial not only for the development process but also for the subsequent evaluation phase. By incorporating real user language and expectations, the chatbot could be designed to better meet user needs and preferences.

A prototype of the chatbot was designed and implemented within the Bosch Smart Home system, utilizing a client-server architecture. The technology stack was carefully selected to include valuable open-source tools.

The evaluation of the chatbot employed a mixed-methods approach, combining quantitative and qualitative analyses. The model performance was assessed primarily through quantitative metrics, but also included a qualitative analysis of successful and unsuccessful outcomes, examining the reasons behind these results. An evaluation dataset was constructed, partly based on the user inputs gathered from the survey, to ensure a comprehensive and realistic assessment of the chatbot's capabilities.

To gain insights into the user experience and overall effectiveness of the chatbot, a user study was conducted. This study comprised a set of tasks for participants to complete using the chatbot, followed by a questionnaire to gather quantitative data on user satisfaction and perceived usability. Additionally, semi-structured interviews were conducted to collect rich, qualitative data on user experiences, preferences, and suggestions for improvement.