% !TeX spellcheck = en_US

\chapter{Concept}
\label{chap:concept}

%% Auslagern in eigenes chapter??? %%
\section{Preleminary Interviews to Gain Knowledge, Requirements and Ideas}
Preceding to further actions Interviews with a few individuals were conducted to gain knowledge, requirements and ideas for this thesis.
All of the four interviewed were from different domains: One is a researcher and has knowledge in \gls{ai} technologies or more precisely about \glspl{llm} while the three other interviewees are from Bosch.
One of them is working in \qq{AskBosch} but also has expertise in the Smart Home and \gls{iot}. 
The other two are both from Bosch Smart Home where one is a Software Developer and the other a Product Developer.
Therefore a diverse group of interviewees has been formed.

The interview was designed to be open but some guided questions were formulated which were selected based on the domain of the interviewed since not all questions met every present domain.
For example it would have not make sense to ask a Smart Home Software Developer about recent tools or the architecture of \gls{llm}-based applications.
In the following all guided questions are listed:

\begin{enumerate}
    \item \textbf{Feasibility and Key Considerations} \\
    What are your thoughts on the feasibility of a smart home chatbot for providing explainability and enhancing user experience?
    \item \textbf{Interesting Intents}
    From your perspective, what could be interesting intents for a smart home chatbot to address complex user queries?    
    \item \textbf{Selected Intents}
    What is your opinion on the specific intents that have been selected for the chatbot development?    
    \item \textbf{Tools/Technologies/Models}
    In your research experience, what tools, technologies, or models would you recommend for developing a chatbot tailored for smart home applications?
    Would you chose a large language model and input context and user request into it or rather a NLP pipeline that chooses further actions based on the user request?   
    \item \textbf{Data Organization}
    How would you suggest organizing and feeding data to the chatbot, considering the complexity of smart home scenarios?    
    \item \textbf{Common Issues/Pitfalls}
    Based on your expertise, what are the common issues or pitfalls researchers may encounter when developing chatbots for specialized domains like smart homes?
    \item \textbf{Evaluation of Success}
    How would you propose evaluating the success of a smart home chatbot, especially in terms of providing explainability and user-centric interaction? 
    \item \textbf{Data Sources for the Chatbot}
    Can you identify potential sources from which data for the chatbot could be extracted to address user queries about the smart home?
    \item \textbf{Accessibility of Data}
    Do you foresee any challenges or limitations in accessing the identified data sources for the chatbot development?
           
\end{enumerate}

\section{Requirements}

\section{Idea}

\section{Architecture}

%\section{further parts of the prototype - NLP Pipeline, Integration into the Bosch SH App,...}

%\blinddocument
